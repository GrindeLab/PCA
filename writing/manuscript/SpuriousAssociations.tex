\documentclass[12pt]{article}

% double-spacing
\usepackage{setspace}
\doublespacing

%% line numbers
%% (turn on before submission)
%\usepackage{lineno}
%\linenumbers

% 1 inch margins
\usepackage[margin=1in]{geometry}

% text color
\usepackage{xcolor}

% bold symbols
\usepackage{bm}

% matrices
\usepackage{amsmath,amsfonts,amssymb}

% images
\usepackage{graphicx}

% links
\usepackage{hyperref}

% remove section from figure number
%\usepackage{chngcntr}
%\counterwithout{figure}{section}

% bibliography
\usepackage[super]{natbib} % superscripts
\renewcommand{\refname}{} % empty title
\setcitestyle{citesep={,}} % comma separated

% highlight text that needs editing in red
\newcommand{\edit}[1]{{\color{red}{#1}}}
\newcommand{\add}[1]{{\color{red}{[... #1 ...]}}}

% reference across files
\usepackage{xr}
\externaldocument{SupplementalInformation}

% keep figures in section
\usepackage[section]{placeins}

\begin{document}

\section*{Target Journal}

\textit{American Journal of Human Genetics}

%\noindent\edit{Other ideas: \textit{PLoS Genetics}, \textit{Genetic Epidemiology}}  %, \textit{Nature Genetics}



\section*{Title}

% no more than 3 lines
% each line <= 54 characters (including spaces)
% convey conceptual significance of paper to broad readership

Adjusting for principal components can induce spurious
 associations in genome-wide association studies
 in admixed populations

\section*{Authors and Affiliations}

Kelsey E. Grinde,$^{1}$* 
Brian L. Browning,$^{2}$ 
Alexander P. Reiner,$^{3,4}$
Timothy A. Thornton,$^{5,6}$ % awaiting confirmation
Sharon R. Browning$^{6}$
%\edit{(middle authors are listed alphabetically by last name)}

\begin{enumerate}
\item Department of Mathematics, Statistics, and Computer Science, Macalester College, Saint Paul, MN, 55105, USA 
\item Division of Medical Genetics, Department of Medicine, University of Washington, Seattle, WA, 98195, USA 
\item Public Health Sciences Division, Fred Hutchinson Cancer Research Center, Seattle, WA, 98109, USA
\item Department of Epidemiology, University of Washington, Seattle, WA, 98195, USA
\item Regeneron Genetics Center, Tarrytown, New York, 10591, USA
\item Department of Biostatistics, University of Washington, Seattle, WA, 98195, USA
\item[*] kgrinde@macalester.edu
\end{enumerate}

%\noindent\edit{Please confirm that your affiliations are correct!} % KG, BB, SB






\newpage
\section*{Abstract}

% single paragraph
% <= 250 words
% convey conceptual advance and significance of work to broad readership
% brief background of question, description of results, brief summary of significance of findings
% do not cite any references

%\edit{Current word count: 246 \\Maximum allowed by AJHG: 250}

\noindent Principal component analysis (PCA) is widely used to control for population structure in genome-wide association studies (GWAS). 
The top principal components (PCs) typically reflect population structure, but deciding how many PCs to include in regression models can be challenging. 
Often researchers err on the side of including more PCs than may be necessary to ensure that population structure is fully captured. 
In this paper, we show that adjusting for extraneous PCs can induce spurious associations, particularly when models include PCs that capture multiple local genomic features (e.g., regions of the genome with atypical linkage disequilibrium (LD)) rather than genome-wide ancestry. 
We investigate the performance of PCA in African American samples from the Women's Health Initiative SNP Health Association Resource and two Trans-Omics for Precision Medicine Whole Genome Sequencing Project studies (the Jackson Heart Study and the Genetic Epidemiology of Chronic Obstructive Pulmonary Disease Study).
We see that PCs are highly correlated with genetic variants across multiple chromosomes, and models adjusting for these PCs yield elevated rates of spurious associations.
LD pruning, using stricter thresholds than is often suggested in the literature, can resolve these issues, whereas excluding high LD regions identified in previous studies does not. 
We also show that the rate of spurious associations is controlled when we simply adjust for the first PC or estimated genome-wide ancestry proportions. 
Our work demonstrates that great care must be taken when using principal components to control for population structure in genome-wide association studies in admixed populations.


\newpage
\section{Introduction}

% succinct
% no subheadings
% present background info necessary to provide biological context for results


%% what is ancestral heterogeneity?
%Admixed populations such as African Americans and Hispanics/Latinos have historically been vastly underrepresented in genome-wide association studies (GWAS) \cite{need2009, bustamante2011, popejoy2016, morales2018, sirugo2019, martin2019}. 
There is considerable variability in global ancestry---the genome-wide proportion of genetic material inherited from each ancestral population---across individuals in admixed populations \citep{parra1998, tishkoff2009, bryc2010aa, bryc2010hl, conomos2016}.
%Considerable variability in global ancestry---the genome-wide proportion of genetic material inherited from each ancestral population---has been observed in studies of admixed populations \citep{parra1998, tishkoff2009, bryc2010aa, bryc2010hl, conomos2016}.
Heterogeneous global ancestry, as with other types of population structure, can lead to spurious associations in genome-wide association studies \citep{GenomicControl, eigenstrat, marchini2004, price2010}. 
In fact, some authors have cited the ancestral heterogeneity of admixed populations, and the statistical challenges it poses, as one reason why these populations have been historically underrepresented in genome-wide association studies (GWAS) \citep{need2009, bustamante2011, popejoy2016, hindorff2018, manolio2019}.
%% why do we need to adjust for it?  (confounding)
Spurious associations can arise in GWAS in ancestrally heterogeneous populations when global ancestry confounds the association between genotypes and the phenotype of interest. % (Figure \ref{fig:confounding}). 
This confounding occurs when the genetic variant being tested differs in frequency across ancestral populations (i.e., global ancestry is associated with genotype) and global ancestry also is correlated with the phenotype of interest via, for example, environmental factors or causal loci elsewhere in the genome that differ in frequency across ancestral groups.

%%%% Figure 1: Confounding DAG %%%
%\begin{figure}
%\center
%\includegraphics[width=0.4\textwidth]{figs/confounding}
%\caption{Global ancestry ($\boldsymbol\pi$) confounds the association between the genotype at position $j$ ($\mathbf{g}_j$) and the phenotype of interest ($\mathbf{y}$) if ancestry is associated with both the genotype (e.g., the allele frequencies differ across the ancestral populations) and the phenotype (e.g., there are environmental or other factors that affect the phenotype and differ across the ancestral populations).}
%\label{fig:confounding}
%\end{figure}


%% how do we adjust
A number of methods for detecting and controlling for ancestral heterogeneity in genetic association studies have been proposed. 
Early approaches included restricting analyses to subsets of ancestrally homogeneous individuals \citep{lander1994}, performing a genome-wide correction for test statistic inflation due to ancestral heterogeneity via genomic control \citep{GenomicControl}, and using family-based designs \citep{tdt}. 
More recently, approaches based on mixed models have been proposed \citep{yu2006, kang2010, yang2014}, using random effects to control for both close (e.g., due to family-based sampling) and distant (e.g., due to shared ancestry) relatedness across individuals.
When studies do not include closely related individuals, a simpler approach is to include inferred global ancestry as a fixed effect in marginal regression models \citep{eigenstrat, pritchard2000}. 
This fixed effects adjustment for global ancestry is used extensively in published studies (e.g., \citep{wellcome2007, fellay2007, novembre2008, reiner2012, carty2012, pino2015, akenroye2021, conti2021}), with global ancestry inferred using either model-based ancestry inference methods or principal component analysis.

% summarize model-based approaches (SKIP --- don't want to emphasize as much)
Model-based approaches for global ancestry inference (e.g., \texttt{frappe} \citep{tang2005}, \texttt{STRUCTURE} \cite{structure}, \texttt{ADMIXTURE} \citep{admixture}, \texttt{RFMix} \citep{rfmix}) model the probability of observed genotypes given unobserved ancestry and allele frequencies in each ancestral population. 
These approaches are used to estimate global ancestry proportions, also known as admixture proportions, which can then be included as covariates in GWAS models to adjust for ancestral heterogeneity.
One of the challenges of using these model-based approaches to infer global ancestry is that the total number of ancestral populations usually needs to be pre-specified. 
In addition, many of these approaches are supervised, requiring reference panel data from each ancestral population of interest to estimate allele frequencies.
Furthermore, ancestry inference is typically conducted at a continental level (e.g., African versus European) so finer levels of population structure could be missed, although some recent efforts have considered global ancestry inference on a sub-continental scale \citep{finestructure, durand2014}.

%% overview of PCA 
Principal component analysis (PCA) is an unsupervised approach for inferring global ancestry that is implemented by a variety of software packages (e.g., \texttt{EIGENSTRAT} \citep{eigenstrat}, \texttt{SNPRelate} \citep{snprelate}, \texttt{PC-AiR} \citep{conomos2015}). 
PCA offers the advantages of not requiring reference panel data or pre-specification of the number of ancestral populations of interest, and it is capable of capturing sub-continental structure \citep{novembre2008}. 
%To infer global ancestry using PCA, we preform a singular value decomposition of the matrix of standardized genotypes (i.e., $\mathbf{X} = \mathbf{U}\mathbf{D}\mathbf{V}^\top$) or, equivalently, an eigenvalue decomposition of the genetic relationship matrix (i.e., $\mathbf{X}\mathbf{X}^\top = \mathbf{U}\mathbf{D}^2\mathbf{U}^\top$).
%where $\mathbf{X}$ is the $n \times m$ matrix of standardized gentoypes for $n$ individuals at $m$ single nucleotide variants (SNVs).
The top principal components (PCs) %$\mathbf{u}_1, \mathbf{u}_2, \dots$ 
tend to reflect global ancestry \citep{patterson2006, mcvean2009}.
To adjust for ancestral heterogeneity in genome-wide association studies, researchers must choose some number of PCs %(typically on the order of 1--10 \citep{abegaz2019}
to include as covariates in GWAS regression models. 
%% issues & solutions: choosing P
Determining this number of PCs needed to capture global ancestry is non-trivial. 
Numerous techniques have been proposed, including formal significance tests based on Tracy-Widom theory \citep{eigenstrat, patterson2006}, examining inflation factors \citep{conomos2016, reed2015} or the proportion of variance explained by each PC \citep{conomos2016, reed2015, raska2012}, comparing PCs to self-reported race/ethnicity \citep{conomos2016}, and keeping PCs that are significantly associated with the trait \citep{reiner2012, daya2019}.
Typically, the number of PCs selected is on the order of one to ten \citep{abegaz2019}, but in practice it is common to see applications in which many more PCs are used---more than may be necessary to capture global ancestry. 
Prior work has suggested that including higher-order PCs can provide the safeguard of removing ``virtually all stratification" \citep{mathieson2012} at the cost of perhaps only ``subtle" decreases in power \citep{liu2011}.
%using the same number of PCs as had been used  to prior work in distinct study populations to justify the choice of 10 PCs \citep{liu2012, reed2015}

%% issues & solutions: other atrifacts 
Another challenge involves ensuring that PCs reflect global ancestry rather than other features or artifacts. 
Principal components can capture relatedness across samples \citep{price2010, conomos2015, patterson2006, abdellaoui2013}, array artifacts or other data quality issues \citep{eigenstrat, price2010, patterson2006,  weale2010}, and small regions of the genome with unusual patterns of linkage disequilibrium (LD) \citep{eigenstrat, price2010, wellcome2007, patterson2006, abdellaoui2013, weale2010,   tian2008, price2008, zou2010, laurie2010, prive2020}. 
%\add{what happens if we adjust for PCs that capture LD instead? include here or save for discussion?}
To address this last issue, many authors have suggested running PCA on a reduced subset of variants after first performing \textit{LD pruning}, using a program such as \texttt{PLINK} \citep{plink} to remove variants that are in ``high" LD (e.g., pairwise-correlation $r^2 > 0.2$) with nearby variants  \citep{conomos2016, wellcome2007,  fellay2007,  novembre2008, conomos2015, reed2015, daya2019,  abdellaoui2013, weale2010, laurie2010, yu2008, nelson2008, anderson2010, zhang2013, galinsky2016}, and/or excluding regions of the genome that are known to have extensive, long-ranging, or otherwise unusual patterns of LD \citep{conomos2016, wellcome2007, fellay2007, novembre2008, raska2012, weale2010,  price2008,  anderson2010}. 
A list of these previously-identified high LD regions and references that recommend their exclusion are provided in Table \ref{tab:highLD}.
%after first removing regions of the genome that are known to have high or long-range LD (see Table \ref{tab:highLD}) \citep{wellcome2007,  fellay2007, novembre2008, price2008, anderson2010, weale2010, raska2012, conomos2016} and/or performing LD pruning (i.e., using a program such as \texttt{PLINK} \citep{plink} to remove variants that are in high LD) \citep{wellcome2007, fellay2007, novembre2008, yu2008, nelson2008, anderson2010, weale2010, laurie2010, abdellaoui2013, zhang2013, conomos2015, reed2015, galinsky2016, conomos2016, daya2019}. 


%\begin{table}
%\center
%\begin{tabular}{crrl}
%Chr & Start (bp) & End (bp) & References \\
%\hline
%1   & 48000000     & 52060567   &    \citep{weale2010, price2008, anderson2010} \\
%2   & 85941853     & 100500000   &    \citep{weale2010, price2008, anderson2010} \\
%2   & 129600000   & 140000000   &    \citep{conomos2016, novembre2008, raska2012, weale2010, price2008,   prive2018} \\
%2   & 182882739    & 190000000   &    \citep{weale2010, price2008, anderson2010} \\
%3   & 47500000     & 50000000    &  \citep{weale2010, price2008, anderson2010} \\
%3   & 83500000     & 87000000    &   \citep{weale2010, price2008, anderson2010} \\
%3   & 89000000     &   97500000   &    \citep{weale2010, price2008} \\
%3   & 163100000    &   164900000   &    \citep{prive2018} \\
%5   & 44000000     &   51500000    &   \citep{fellay2007, weale2010, price2008, anderson2010} \\
%5   & 98000000     &  100500000   &    \citep{weale2010, price2008} \\
%5   & 129000000    &   132000000   &    \citep{weale2010, price2008, anderson2010} \\
%5   & 135500000    &   138500000   &   \citep{weale2010, price2008} \\
%6   &  23800000     &   39000000   &    \citep{conomos2016, fellay2007, novembre2008, raska2012, weale2010,  price2008, anderson2010, prive2018} \\
%6   &  57000000     &   64000000   &    \citep{weale2010, price2008, anderson2010}  \\
%6   & 140000000    &   142500000   &    \citep{weale2010, price2008, anderson2010}  \\
%7   &  55000000     &   66193285   &    \citep{weale2010, price2008, anderson2010}  \\
%8   &   6300000      &  13500000   &    \citep{conomos2016, fellay2007, novembre2008, raska2012, weale2010, tian2008,  price2008, anderson2010,  prive2018} \\
%8   &  43000000    &    50000000   &    \citep{weale2010, price2008, anderson2010}  \\
%8   & 112000000    &   115000000    &  \citep{weale2010, price2008, anderson2010} \\
%10  &  37000000    &    43000000   &   \citep{weale2010, price2008, anderson2010}  \\
%11   & 45000000    &    57000000   &    \citep{fellay2007, weale2010, price2008} \\
%11   & 87500000    &    90500000   &    \citep{weale2010, price2008, anderson2010}  \\
%12   & 33000000    &    40000000   &    \citep{weale2010, price2008, anderson2010} \\
%12   & 109500000   &    112021663   &    \citep{weale2010, price2008} \\
%14   & 46600000    &    47500000   &    \citep{prive2018} \\
%17   & 37800000    &    42000000    &   \citep{conomos2016, novembre2008} \\
%20   & 32000000   &     34500000    &   \citep{weale2010, price2008, anderson2010}  \\
%\end{tabular}
%\caption{Regions of the genome with high, long-range, or otherwise unusual linkage disequilibrium that are often recommended for exclusion prior to running PCA. Start and end physical (base pair) positions are provided with respect to genome build 36. This list is also available for download in builds 36, 37, and 38 at \href{github.com/kegrinde/PCA}{https://github.com/kegrinde/PCA/}.}
%\label{tab:highLD}
%\end{table}

\begin{table}
\center
\begin{tabular}{crrl}
Chr & Start (bp) & End (bp) & References \\
\hline
1   & 47761741     & 51822307   &    \citep{weale2010, price2008, anderson2010} \\
%2   & 85941853     & 100500000   &    \citep{weale2010, price2008, anderson2010} \\
2   & 129125957   & 139525961   &    \citep{conomos2016, novembre2008, raska2012, weale2010, price2008,   prive2018} \\
2   & 182309767    & 189427029   &    \citep{weale2010, price2008, anderson2010} \\
3   & 47483506     & 49987563    &  \citep{weale2010, price2008, anderson2010} \\
3   & 83368159     & 86868160    &   \citep{weale2010, price2008, anderson2010} \\
%3   & 89000000     &   97500000   &    \citep{weale2010, price2008} \\
3   & 161899518    &   163699518   &    \citep{prive2018} \\
%5   & 44000000     &   51500000    &   \citep{fellay2007, weale2010, price2008, anderson2010} \\
5   & 98636396     &  101136397   &    \citep{weale2010, price2008} \\
5   & 129636408    &   132636409   &    \citep{weale2010, price2008, anderson2010} \\
5   & 136136412    &   139136412   &   \citep{weale2010, price2008} \\
6   &  23691793     &   38924246   &    \citep{conomos2016, fellay2007, novembre2008, raska2012, weale2010,  price2008, anderson2010, prive2018} \\
%6   &  57000000     &   64000000   &    \citep{weale2010, price2008, anderson2010}  \\
6   & 139637170    &   142137170   &    \citep{weale2010, price2008, anderson2010}  \\
%7   &  55000000     &   66193285   &    \citep{weale2010, price2008, anderson2010}  \\
8   &   6455071      &  13598120   &    \citep{conomos2016, fellay2007, novembre2008, raska2012, weale2010, tian2008,  price2008, anderson2010,  prive2018} \\
%8   &  43000000    &    50000000   &    \citep{weale2010, price2008, anderson2010}  \\
8   & 110918595    &   113918595    &  \citep{weale2010, price2008, anderson2010} \\
%10  &  37000000    &    43000000   &   \citep{weale2010, price2008, anderson2010}  \\
%11   & 45000000    &    57000000   &    \citep{fellay2007, weale2010, price2008} \\
11   & 88127184    &    91127184   &    \citep{weale2010, price2008, anderson2010}  \\
%12   & 33000000    &    40000000   &    \citep{weale2010, price2008, anderson2010} \\
12   & 110577812   &    113099475   &    \citep{weale2010, price2008} \\
14   & 47061047    &    47961047   &    \citep{prive2018} \\
17   & 42394456    &    46567318    &   \citep{conomos2016, novembre2008} \\
20   & 33948533   &     36438183    &   \citep{weale2010, price2008, anderson2010}  \\
\end{tabular}
\caption{Regions of the genome with high, long-range, or otherwise unusual linkage disequilibrium that are often recommended for exclusion prior to running PCA. Start and end physical (base pair) positions are provided with respect to genome build 38. This list is also available for download in builds 36, 37, and 38 at \href{github.com/kegrinde/PCA}{https://github.com/kegrinde/PCA/}.}
\label{tab:highLD}
\end{table}


%% motivate  our paper
LD pruning and filtering are not universally practiced, and the downstream implications of adjusting for PCs that capture features other than global ancestry are not fully understood.
Furthermore, much of this prior work was conducted in populations of European ancestry, so recommendations on how best to implement principal component-based adjustment for ancestral heterogeneity in admixed populations are lacking. 
% outline rest of paper
In this paper, we investigate the impact of LD filtering and pruning choices, as well as choices of the number of principal components to include in analyses, on genome-wide association studies in admixed populations.
We show that principal components may capture local genomic features, unless careful pre-processing is performed prior to analysis.
We also conduct simulation studies and provide analytic results to show that including too many PCs can induce spurious associations in GWAS, particularly when those extraneous PCs capture local genomic features rather than genome-wide ancestry.
To conclude, we provide suggestions regarding best practices for  controlling for ancestral heterogeneity in genome-wide association studies in admixed populations.



\section{Material and Methods}

% provide sufficient detail so readers can understand how experiments were performed and procedures can be repeated
% describe any statistical methods employed in study

\subsection{Data and Quality Control}

Our analyses focus on array-based and whole genome sequence-based genotype data from three samples of African American individuals.
In particular, we consider genotype data from the Women's Health Initiative SNP Health Association Resource (WHI SHARe), as well as whole genome sequencing data from two contributing studies to the Trans-Omics for Precision Medicine (TOPMed) Whole Genome Sequencing Project: the Jackson Heart Study (JHS) and the Genetic Epidemiology of Chronic Obstructive Pulmonary Disease Study (COPDGene).
We performed quality control and identified subsets of unrelated African American individuals prior to running any further analyses---details are provided below.

\subsubsection{WHI SHARe Genotype Data}

The Women's Health Initiative (WHI) is a long-term study of the health of post-menopausal women residing in the United States.
In total, 161,808 women aged 50--79 years old were recruited to participate in this study.
Additional details of the study design and cohort characteristics can be found elsewhere \cite{whi}.
The WHI SHARe study includes 12,151 self-identified African American women who consented to genetic research, a subsample of which were selected for genotyping using the Affymetrix Genome-Wide Human SNP Array 6.0.
This array contains 906,000 single nucleotide polymorphisms (SNPs) and more than 946,000 probes for detection of copy number variants. 
In our analyses, we focus only on the SNP data. 
We did not impute WHI genotypes beyond filling in sporadic missing genotypes---see Grinde et al. \cite{steam} for more details.

The genotype data were processed for quality control.
After filtering on call rate, concordance rates for blinded and unblinded duplicates, and sex discrepancy, there were 871,309 SNPs with a missing genotyping rate of 0.2\% and 8,421 African American women \cite{reiner2012}.
We also used the iterative procedure suggested by Conomos et al. \cite{conomos2016related} to identify a subset of 8,064 mutually unrelated individuals, using a kinship threshold of 0.044 (i.e., excluding first-, second-, and third-degree relatives).

\subsubsection{TOPMed Whole Genome Sequence Data}

The TOPMed Whole Genome Sequencing Project is an ongoing project sponsored by the National Heart, Lung, and Blood Institute.
The goal of this project is to collect and analyze whole-genome sequences, other omics data, and extensive phenotypic information for over 100,000 individuals from diverse backgrounds. 
Data are periodically released on dbGaP for analysis by the broader scientific community. 
Our analysis uses data from freeze 4, released in 2017, and freeze 5b, released in 2018.
These two freezes include samples from a large number of contributing studies.
We focus on two of these studies: the Jackson Heart Study (JHS) (accession number: phs000964) and the Genetic Epidemiology of Chronic Obstructive Pulmonary Disease Study (COPDGene) (accession number: phs000951).
% sequencing
High coverage ($\approx$ 30X) whole genome sequencing was performed by the University of Washington (UW) Northwest Genomics Center for JHS (freeze 4) and by a combination of the UW Northwest Genomics Center and the Broad Institute of MIT and Harvard for freeze 5b COPDGene.
%Variant discovery and genotype calling was performed by the TOPMed Informatics Resources Center (IRC) using the \texttt{GotCloud} pipeline \citep{jun2015}.
% QC
%Quality control (QC) was performed by the sequencing centers, IRC, and TOPMed Data Coordinating Center, and only those samples and variants that passed these stages of QC are included in the Variant Call Format (VCF) files downloaded from dbGaP.
% references
Details on TOPMed sequencing and QC methods are available in Taliun et al. \citep{taliun2021} and on the TOPMed website: \href{https://topmed.nhlbi.nih.gov/data-sets}{https://topmed.nhlbi.nih.gov/data-sets}.
% sample sizes
In total, the downloaded freeze 4 JHS dataset includes 2,777 African American individuals and the freeze 5b COPDGene dataset includes 8,476 African American and European American individuals consented for biomedical research.

%% filters applied universally
Prior to genetic ancestry inference, we performed two additional stages of variant- and sample-level filtering.
% bcftools filters from Joe
We used \texttt{bcftools} \citep{bcftools} to restrict our analyses to biallelic single nucleotide variants (SNVs). 
% relatives 
To identify a subset of mutually unrelated individuals (kinship threshold = 0.044), we used the UW Genetic Analysis Center (GAC) TOPMed analysis pipeline.
In addition to inferring relatedness using the procedure proposed by Conomos et al. \cite{conomos2016related}, this pipeline also includes code to perform PCA, association testing, and other tasks in whole genome sequence data: more details can be found at \href{https://github.com/UW-GAC/analysis_pipeline}{https://github.com/UW-GAC/analysis\_pipeline}.
%which implements the Conomos et al. iterative procedure \cite{conomos2016related} on sequence data, to restrict our analyses to a subset of mutually unrelated individuals (kinship threshold = 0.044). 
%(Will this confuse people? Is saying we used the TOPMed pipeline enough? Perhaps skip?:) Note that this procedure of identifying related individuals involves an initial step of LD pruning and filtering, as recommend by the TOPMed Analysis Pipeline.
After variant filtering and removal of related individuals, 1,928 and 8,406 unrelated samples and 77,136,850 and 135,522,041 variants remained in JHS and COPDGene, respectively.


\subsection{Genetic Ancestry Inference}

We consider two approaches to inferring genetic ancestry in these admixed samples: model-based approaches and principal component analysis.

\subsubsection{Model-Based Approaches}

In WHI SHARe African Americans, we inferred both local and global genetic ancestry using model-based ancestry inference techniques. 
Local ancestry inference was performed using \texttt{RFMix} \citep{rfmix} and a reference panel including individuals of European and African descent from the International HapMap Project (HapMap) \citep{hapmap}: see Grinde et al. \cite{steam} for more details.
We then calculated global ancestry proportions via the genome-wide average local ancestry $\hat\pi_{ik} = \frac{1}{2m}\sum_{j=1}^m a_{ijk},$ where $a_{ijk}$ is the inferred number of alleles (0, 1, or 2) inherited by individual $i$ at variant $j$ from ancestral population $k$.
We also compared these \texttt{RFMix}-based global ancestry estimates to results from supervised and unsupervised \texttt{ADMIXTURE}  \citep{admixture} analyses with two ancestral populations ($K = 2$).
The supervised analysis used the same HapMap reference panel as was used to infer local ancestry using \texttt{RFMix}.
% supervised = same Hap
All three sets of admixture proportions were highly correlated (pairwise Pearson correlation $>$ 0.998). 
We proceed with using only the \texttt{RFMix}-based admixture proportion estimates for the remainder of our analyses.

In TOPMed JHS and COPDGene samples, we inferred global ancestry via unsupervised \texttt{ADMIXTURE} analyses with both two ancestral populations. 
% did not do supervised or RFMix because more time-intensive
We also used these inferred global ancestry proportions to identify subsets of admixed individuals.
Although JHS recruited only self-identified African Americans, we did identify 40 individuals in the sample with an estimated European ancestry proportion of 100\%.
These individuals were excluded from further analyses, leaving a total of 1,888 unrelated admixed samples. 
The COPDGene study, by design, includes both African American and European American individuals.
Self-identified race/ethnicity information was not available from dbGaP, so we used inferred admixture proportions to identify and restrict our attention to individuals with at least 29.5\% African ancestry.
% although these analyses were unsupervised, we inferred which ancestral population corresonded to African ancestry based on observed ancestry proportions
The choice of threshold follows from the results reported by Parker et al. \cite{parker2014}, showing that the self-identified African American individuals in the COPDGene study have inferred proportions of African ancestry ranging from 29.5\% and above.
(We are not suggesting that this same threshold be universally applied to identify African American individuals in other samples.)
After filtering, 2,676 individuals remain in COPDGene. 
The remainder of our TOPMed analyses included only these subsets of unrelated admixed individuals.



\subsubsection{Principal Component Analysis}

%% mathematical details of PCA (moved from intro)
To infer global ancestry using PCA, we performed a singular value decomposition of the matrix of standardized genotypes (i.e., $\mathbf{X} = \mathbf{U}\mathbf{D}\mathbf{V}^T$) or, equivalently, an eigenvalue decomposition of the genetic relationship matrix (i.e., $\mathbf{X}\mathbf{X}^T = \mathbf{U}\mathbf{D}^2\mathbf{U}^T$), where $\mathbf{X}$ is the $n \times m$ matrix of standardized genotypes for $n$ individuals at $m$ single nucleotide variants.
One or more of the top eigenvectors, or principal components, $\mathbf{u}_1, \mathbf{u}_2, \dots$, typically reflect global ancestry.

%% PCA in WHI
We ran PCA on the WHI SHARe genotype data using \texttt{SNPRelate} \citep{snprelate}. 
First, we applied PCA to the same set of 551,025 SNPs used to estimate global ancestry proportions.
%We refer to these PCs as the \textit{naively generated PCs}.
We then applied PCA to subsets of SNPs based on the following pre-processing criteria: excluding SNPs falling into regions of the genome that have been cited in the literature as potentially problematic for PCA (Table \ref{tab:highLD}), LD pruning, or both literature-based exclusions and LD pruning.
To perform LD pruning, two parameters must be specified: $r^2$ threshold and window size. 
Here we use an $r^2$ threshold of 0.1 and window size of 0.5 mega basepairs (Mb), which is stricter than is often suggested in the literature: for a full discussion of these choices, see Supplemental Information Section \ref{sec:comparefilters}.
Both LD pruning and filtering of regions in Table \ref{tab:highLD} were implemented using the \texttt{SNPRelate} package.
Table \ref{tab:preprocessN} summarizes the number of SNPs remaining after each set of pre-processing steps.
After running PCA on these different sets of variants, we also used \texttt{SNPRelate} to assess the contribution of each SNP to each principal component by calculating the SNP loadings and the correlation between PCs and genotypes.

%% PCA in TOPMed
In TOPMed JHS and COPDGene samples, we used the UW GAC TOPMed analysis pipeline to implement pre-processing, run principal component analysis, and calculate and visualize the contribution of individual variants to each PC. 
Similar to WHI SHARe, we applied PCA to various subsets of variants based on different pre-processing criteria, including a naive analysis with no prior LD-based pruning or filtering, an analysis after excluding the regions listed in Table \ref{tab:highLD}, an analysis after LD pruning with an $r^2$ threshold of 0.1, and an analysis after both LD-based pruning and filtering.
Following the recommendations of Kirk \citep{JKdissertation} and the UW GAC pipeline documentation, we also removed variants with a minor allele frequency lower than 0.01 in all cases.
Note that the UW GAC pipeline provides the option to exclude some of the regions listed in Table \ref{tab:highLD} (the \textit{LCT} gene on chromosome 2, the HLA region on chromosome 6, and the locations of large inversions on chromosomes 8 and 17), but we customized the pipeline code to add the other regions identified in our literature review. 
See Table \ref{tab:preprocessN} for the number of variants included in each analysis.

\begin{table}
\small
\begin{tabular}{|l|rrr|}
\hline
 & WHI SHARe & TOPMed JHS & TOPMed COPDGene \\
\hline
Neither Exclude nor Prune & 551,025 & 14,117,957 & 13,959,378 \\
Exclude High LD Regions (Table \ref{tab:highLD}) & 536,668 & 13,723,944 & 13,575,038  \\
LD Prune ($r^2 < 0.1$, 0.5 Mb windows) & 49,723  &  245,003 & 251,040  \\
Both Exclude and Prune & 48,794 & 239,425  & 245,378  \\
\hline
\end{tabular}
\caption{Number of autosomal SNPs remaining after different combinations of pre-processing steps were applied prior to running PCA. Note that TOPMed analyses include only those variants with minor allele frequency greater than 1\%.} 
\label{tab:preprocessN}
\end{table}

%\begin{table}
%\begin{tabular}{ccccc}
%name & chrom & start.base  & end.base  & comment \\
%2q21    &  2  &  129125957 & 139525961       &  LCT \\
%HLA      &    6 &   24091793  & 38924246 & includes MHC \\
%8p23  &        8   &  6755071 &  13598120 &   inversion \\
%17q21.31 &    17  &  42394456 & 46567318   &  inversion \\
%\end{tabular}
%\caption{TOPMed hg38 high corr regions}
%\end{table}



\subsection{Simulation Study}

To explore the impact of adjusting for principal components that capture local genomic features, we conducted a simulation study using genotype data and simulated traits in the WHI SHARe African American sample.

\subsubsection{Trait Simulation}

Traits were simulated for each individual $i = 1, \dots, 8064$ such that they depended only on the genotype $g_{ij}$ at a single causal variant with effect size $\beta_j$: $$y_i = \beta_j g_{ij} + \epsilon_i, \ \epsilon_i \stackrel{iid}{\sim} N(0, 1).$$
We considered seven choices of effect sizes ($\beta_j = 0, 0.25, 0.5, 1, 2, 4, 8$) and nearly 500 choices for the position $j$ of the causal variant, varying the position of this causal variant across all 22 autosomes.

To choose the location of these causal variants, we first estimated the difference in ancestral allele frequencies for each variant using the observed allele frequencies in our HapMap reference panel which included samples from the CEU (Utah residents with Northern and Western European ancestry) and YRI (Yoruba in Ibadan, Nigeria)  populations: see Grinde et al. \citep{steam} for more details.
We also considered the SNP loadings %(i.e., the contribution of each SNP to each PC \add{\cite{136}}) 
for the set of PCs that were generated without any prior LD-based filtering or pruning.
For each of the first four PCs, we identified the 220 variants (10 per chromosome) with the highest absolute SNP loadings. 
Many of these variants with large loadings for one PC also had large loadings for another PC, so in total there were 373 unique variants selected according to 	this procedure. 
For comparison, we also selected 100 variants across the autosomes with low SNP loadings ($|\text{loading}| < 0.0008$) for all of the first four PCs.
Among these 100 variants, 85 were selected such that they had different allele frequencies in the African and European ancestral populations ($|\hat{p}_{CEU} - \hat{p}_{YRI}| > 0.6$), and 15 were selected that had similar allele frequencies in the two ancestral populations ($|\hat{p}_{CEU} - \hat{p}_{YRI}| < 0.005$).
Altogether, we selected 473 variants with positions spread across the genome, high or low SNP loadings, and large or small ancestral allele frequency differences to investigate the impact of different characteristics of the causal variant on GWAS results.


\subsubsection{GWAS Models}

For each simulated trait, we ran genome-wide association studies using models of the general form
$$E[y_i \mid g_{ij}, \mathbf{w}_i] = \alpha + \beta_j g_{ij} + \boldsymbol\gamma \mathbf{w}_i,$$
where $y_i$ is the simulated quantitative trait, $g_{ij}$ is the genotype at position $j$,  and $\mathbf{w}_i$ is a vector of additional covariates.
Note that we quantify genotype $g_{ij}$ by the number of copies---0, 1, or, 2---of some pre-specified allele (e.g., the minor allele) carried by individual $i$ at position $j$.
We fit these models at every position $j = 1, \dots, m$ across the genome and test for association between the trait and genotype by testing the null hypothesis $H_0: \beta_j = 0$ using a traditional Wald test.

In particular, we consider four models: a model making no adjustment for ancestral heterogeneity (i.e., $\mathbf{w}_i = \emptyset$), a model adjusting for estimated admixture proportions ($\mathbf{w}_i = \hat\pi_i$), a model adjusting for the first principal component ($\mathbf{w}_i = u_{1i}$), and a model adjusting for the first four principal components ($\mathbf{w}_i = \begin{bmatrix} u_{1i} & u_{2i} & u_{3i} & u_{4i} \end{bmatrix}$). 
For the models adjusting for principal components, we consider four sets of PCs based on different pre-processing criteria: \textit{none} (no prior LD-based exclusions or pruning), \textit{exclusions only} (excluding regions from Table \ref{tab:highLD} but no LD pruning), \textit{pruning only} (LD pruning with $r^2 < 0.1$ and a window size of 0.5 Mb, but not excluding regions from Table \ref{tab:highLD}), and \textit{both} (both Table \ref{tab:highLD} exclusions and LD pruning).

\subsubsection{Spurious Associations}

To evaluate these ancestral heterogeneity adjustment approaches, we compared the number of spurious associations that appeared when we used each model.
We quantified spurious associations by counting the number of chromosomes, not including the chromosome on which the causal variant was located, with at least one variant reaching genome-wide significance.
For all models, the genome-wide significance threshold was set to the $p = 5.0 \times 10^{-8}$ threshold that is used extensively throughout the GWAS literature \cite{peer2008, jannot2015}.


%old outline: 
%
%\begin{itemize}
%\item TOPMed data
	%\begin{itemize}
	%\item what is TOPMed
	%\item which studies did we focus on
	%\item how accessed
	%\item who's in it
	%\item freeze 4 sequencing methods (JHS)
	%\item freeze 5b methods (COPDGene)
	%\item TOPMed QC
	%\end{itemize}
%\item filtering
%	\begin{itemize}
	%\item basic SNP filters --> moved to TOPMed section
	%\item rare variants \add{move this to QC step?}; \add{citations}
	%\item missing rates \add{STILL NEED TO IMPLEMENT THIS!!}
	%\item relatives; \add{citations}
	%\item non-admixed individuals
	%\item LD pruning/filtering
	%\end{itemize}
%\item inferring ancestry using PCA
	%\begin{itemize}
	%\item software
	%\item types of pruning/filtering considered --> move to filtering
	%\item plots we look at (loadings, screeplots, parallel coordinates, etc.)
	%\end{itemize}
%\item inferring ancestry using ADMIXTURE --- skip since already mentioned in QC
	%\begin{itemize}
	%\item motivation for comparison
	%\item recommended pruning/filtering
	%\end{itemize}
%\item simulation study
%	\begin{itemize}
%	\item traits
%	\item models
%	\item evaluation
%	\end{itemize}
%\item software and data availability
	%\begin{itemize}
	%\item dbgap
	%\item github
	%\end{itemize}
%\end{itemize}


%\subsection{Old Methods}

%\textbf{QC for JHS} (dbgap accession phs000964):
%
%\begin{itemize}
%\item filtering
	%\begin{itemize}
	%\item bi-allelic SNPs
	%\item minor allele count at least 1
	%\item pass variant filters (in VCFs that were downloaded from dbgap): overlaps with SNP, overlaps with indel, overlaps with VNTR, failed SVM filter, high (3/5\% or more) mendelian or duplicate genotype discordance, excess heterozygosity with HWE p-value < 1e-6
	%\end{itemize}
%\item merge the two subsets (cg1 and cg3)
%\item convert from VCF to GDS
%\item remove close relatives
	%\begin{itemize}
	%\item run king (LD r threshold: 0.32, LD window size: 10, MAF threshold: 0.01, exclude PCA corr: TRUE, build: hg19)
	%\item run PC-AiR 
	%\item run PCRelate 
	%\item run PC-AiR again
	%\end{itemize}
%\item find African Americans
	%\begin{itemize}
	%\item run stricter LD pruning (MAF  = 0.01, window size = 0.5, rsq = 0.01, regions = TRUE, build 37)
		%\begin{itemize}
		%\item List of regions stored here: \verb"/projects/browning/brwnlab/kelsey/spurious_assoc/highLD_regions/" (see below)
		%\end{itemize}
	%\item convert GDS to BED
	%\item run ADMIXTURE (K = 2 and K = 3)
	%\item plot proportions
	%\item exclude 40 people inferred to be 100\% European; left with 1888
	%\end{itemize}
%\end{itemize}

%\textbf{QC for COPDGene} (phs000951):
%
%\begin{itemize}
%\item filtering
%	\begin{itemize}
%	\item bi-allelic SNPs
%	\item minor allele count at least 1
%	\item pass filtering (from GDS annotation info, I inferred this to include: variant located in centromeric region, variant failed SVM filter, mendelian or duplicate genotype discordance is high (3/5\% or more), excess heterozygosity in chrX in males, excess heterozygosity with HWE p-value < 1e-6)
%	\end{itemize}
%item convert from VCF to GDS
%\item remove close relatives
%	\begin{itemize}
%	\item run king (LD r threshod = 0.32, LD window size = 10, MAF threshodl = 0.01, exclude PCA corr = TRUE, build = hg38); regions =         see table below
%	\item run PCAiR
%	\item run PCRelate
%	\item run PCAiR again
%	\end{itemize}
%\item find African Americans
%	\begin{itemize}
%	\item run stricter LD pruning (exclude PCA corr regions = TRUE, build = hg38, LD R threshold = 0.1, LD window size = 10, MAF threshold = 0.01); regions = \verb"/projects/browning/brwnlab/kelsey/spurious_assoc/highLD_regions/" (see below)
%	\item convert from GDS to PLINK
%	\item run ADMIXTURE with K = 2 and K = 3
%	\item plot proportions and use cut-off of 30\% to identify (and then remove) Europeans : Parker et al. 2014 "Admixture mapping identifies a quantitative trait locus associated with..." \verb"https://www.ncbi.nlm.nih.gov/pmc/articles/PMC4190160/" (reduced from 8406 to 2676)
%	\end{itemize}
%\end{itemize}

%\begin{table}
%\begin{tabular}{ccccc}
%name & chrom & start.base  & end.base  & comment \\
%2q21    &  2  &  129883530 & 140283530       &  LCT \\
%HLA      &    6 &   24092021  & 38892022 & includes MHC \\
%8p23  &        8   &  6612592 &  13455629 &   inversion \\
%17q21.31 &    17  &  40546474 & 44644684   &  inversion \\
%\end{tabular}
%\caption{TOPMed hg19 high corr regions}
%\end{table}

%\begin{table}
%\begin{tabular}{ccccc}
%name & chrom & start.base  & end.base  & comment \\
%2q21    &  2  &  129125957 & 139525961       &  LCT \\
%HLA      &    6 &   24091793  & 38924246 & includes MHC \\
%8p23  &        8   &  6755071 &  13598120 &   inversion \\
%17q21.31 &    17  &  42394456 & 46567318   &  inversion \\
%\end{tabular}
%\caption{TOPMed hg38 high corr regions}
%\end{table}

%\begin{itemize}
%\item what filtering was performed, and how many variants left after filtering
%	\begin{itemize}
%	\item JHS, ADMIXTURE: see above
%	\item JHS, PCA: exclude regions (TRUE/FALSE), r-squared (1, 0.1, 0.2, 0.05), window size (0, 0.5, 10), and MAF (0, 0.01) 
%		\begin{itemize}
%		\item no filtering: FALSE-1-0-0
%		\item MAF filtering: FALSE-1-0-0.01
%		\item exclude but no prune: TRUE-1-0-0.01
%		\item prune but no exclude: FALSE-0.1-0.5-0.01 and FALSE-0.1-10-0.01 and FALSE-0.2-0.5-0.01 and FALSE-0.05-0.5-0.01
%		\item prune and exclude: TRUE-0.1-0.5-0.01 and TRUE-0.1-10-0.01 and TRUE-0.2-0.5-0.01 and TRUE-0.05-0.5-0.01
%		\end{itemize}
%	\item COPD, ADMIXTURE: see above
%	\item COPD, PCA: exclude regions (TRUE/FALSE), r-squared (1, 0.1, 0.2, 0.05), window size (0, 0.5, 10), MAF (0, 0.01)
%		\begin{itemize}
%		\item no filtering: FALSE-1-0-0
%		\item MAF filtering: FALSE-1-0-0.01
%		\item exclude but no prune: TRUE-1-0-0.01
%		\item prune but no exclude: FALSE-0.1-0.5-0.01, FALSE-0.1-10-0.01, FALSE-0.2-0.5-0.01, FALSE-0.05-0.5-0.01
%		\item prune and exclude: TRUE-0.1-0.5-0.01, TRUE-0.1-10-0.01, TRUE-0.05-0.5-0.01, TRUE-0.2-0.5-0.01
%		\end{itemize}	
%	\item COPD, also ran SNPRelate on Europeans with different levels of filtering (FALSE-0.1-0.5-0.01, FALSE-0.2-0.5-0.01, FALSE-1-0-0.01, FALSE-1-0-0, TRUE-0.1-0.5-0.01, TRUE-0.2-0.5-0.01, TRUE-1-0-0.01)
%	\end{itemize}
%\end{itemize}


%\subsection{Simulation Study to Investigate Rates of Spurious Associations}
%
%We implement a simulation study to explore the impact of different variant-level filtering choices, particularly with respect to linkage disequlibrium, on rates of spurious associations in genome-wide association studies using models that adjust for ancestral heterogeneity using principal components.
%
%\subsubsection{Simulated Traits}
%
%\add{
%\begin{itemize}
%\item find loading peaks from "naive" approach
%\item simulate trait that is beta * x + rnorm(0, 1), where beta = 1 or 2 and x = genotype at one of the peaks
%\end{itemize}
%}

%\subsubsection{GWAS Models}
%
%To perform genome-wide association studies in samples of unrelated admixed individuals, we use marginal regression models, regressing the trait of interest on the genotype at each position across the genome. 
%At a given position $j$, we quantify genotype $g_{ij}$ as the number of copies (0, 1, or 2) of some pre-specified allele (e.g., the minor allele) carried by individual $i$ at that position. 
%Considering a quantitative trait $y_i$, we fit one linear regression model at each position ($j = 1, \dots, m$): $$E[y_i \mid g_{ij}, \mathbf{z}_i] = \beta_0 + \beta_j g_{ij} + \boldsymbol{\beta}_z \mathbf{z}_i,$$ where $\mathbf{z}_i$ is a vector of additional covariates (e.g., potential confounding variables) that we want to include in the model.
%%This linear regression model can be replaced with a logistic regression model in the case of a binary trait (e.g., disease status).
%We test for an association between the trait and genotype by testing the null hypothesis $H_0: \beta_j = 0$ at each position $j = 1, \dots, m$.
%
%
%To adjust for ancestral heterogeneity, we include inferred global ancestry in the vector $\mathbf{z}_i$ of potential confounders in our regression models. We infer global ancestry using one of two techniques: model-based global ancestry inference of principal component analysis. 
%\add{describe which models we compare}
%
%\add{
%\begin{itemize}
%\item for each of 188*2 simulated phenotypes
%\item for each set of PCs
%\item including 1, 4, or 10 PCs
%\end{itemize}
%}

%% UPDATE: skipping details of model-based GAI
%\subsubsection{Model-based global ancestry inference}
%
%Various model-based approaches have been developed for estimating global ancestry proportions in admixed populations.
%We represent global ancestry via the vector $\boldsymbol{\pi}_i = \begin{pmatrix} \pi_{i1} & \dots & \pi_{iK} \end{pmatrix}^\top$, where $\pi_{ik}$ denotes the genome-wide proportion of genetic material inherited by individual $i$ from ancestral population $k$ and $\sum_{k=1}^K \pi_{ik} = 1$. 
%Note that the total number of ancestral populations, $K$, typically must be pre-specified, and the definition of global ancestry is typically restricted to the autosomes.
%Admixture proportions can be estimated directly using a program such as \texttt{ADMIXTURE} \citep{admixture}, or by calculating the genome-wide average local ancestry (i.e., $\hat\pi_{ik} = \frac{1}{2m} \sum_{j=1}^m a_{ijk}$), where local ancestry $a_{ijk}$---the number of alleles (0, 1, or 2) inherited by individual $i$ from ancestral population $k$ at position $j$---was first inferred using a program such as \texttt{RFMix} \citep{rfmix}.
%To adjust for ancestral heterogeneity, we include $K-1$ of these estimated admixture proportions as covariates in our GWAS regression models: $$E[y_i \mid g_{ij}, \hat{\boldsymbol\pi}_i] = \beta_0 + \beta_j g_{ij} + \beta_{\pi, 1} \hat\pi_{i,1} + \dots + \beta_{\pi, K-1} \hat\pi_{i, K-1}.$$
%
%Many model-based global ancestry inference programs are supervised, requiring data from individuals from each ancestral population of interest to form a reference panel. However, some approaches such as \texttt{ADMIXTURE} can also be run without a reference panel.

%% UPDATE: moved most of this to intro
%\subsubsection{Principal component analysis}
%
%Principal component analysis (PCA) is an unsupervised dimension-reduction technique that is widely used for inferring population structure in genetic studies, with a number of software programs available for running PCA on genotype or sequence data (e.g., \texttt{EIGENSTRAT} \citep{eigenstrat}, \texttt{SNPRelate} \citep{snprelate}, \texttt{PC-Air} \citep{conomos2015}).
%To run PCA, we perform a singular value decomposition of the matrix of standardized genotypes (i.e., $\mathbf{X} = \mathbf{UDV}^\top$) or, equivalently, an eigenvalue decomposition of the genetic relationship matrix (i.e., $\mathbf{XX}^\top = \mathbf{UD}^2\mathbf{U}^\top$). 
%The top principal components ($\mathbf{u}_1, \mathbf{u}_2, \dots$) typically capture global ancestry \citep{patterson2006, mcvean2009}. 
%To adjust for ancestral heterogeneity, we choose some number of principal components, $P$, needed to capture global ancestry (typically $1 \le P << n$) and include those PCs as covariates in our GWAS regression models: $$E[y_i \mid g_{ij}, u_{i1}, \dots, u_{iP}] = \beta_0 + \beta_j g_{ij} + \beta_{u 1} u_{i1} + \dots + \beta_{u P} u_{i P}.$$
%A number of techniques have been proposed for selecting the number of PCs, $P$, including formal significance tests based on Tracy-Widom theory \citep{patterson2006, eigenstrat}, examining the proportion of variance explained by each PC \citep{reed2015}, comparing PCs to self-reported ancestry \citep{conomos2016}, and/or keeping PCs that are significantly associated with the trait \citep{reiner2012, daya2019}. 



%% UPDATE: moved most of this to intro
%\subsubsection{Variant- and sample-level filtering}
%It is often recommended that filtering be performed at the variant and/or sample level prior to inferring global ancestry. 
%Prior work has shown that both model-based estimates of global ancestry \add{cite Tim's GAW paper} and principal components \citep{conomos2015, eigenstrat, patterson2006} \add{check patterson, maybe add more refs: Price 2010} can reflect family structure and/or cryptic relatedness rather than global ancestry when a sample includes related individuals, but restricting analyses to a subset of unrelated individuals (e.g., using the iterative procedure proposed by \cite{conomos2016related}) can circumvent that issue. 
%At the variant level, it is common to perform filtering based on minor allele frequency \add{find references}, as prior work has shown that methods such as \texttt{EIGENSTRAT} can perform poorly when applied to rare variants \citep{kirk2016}.
%
%Other variant-level filters have been recommended to address the sensitivity of model-based and PCA approaches to the presence of linkage disequilibrium (LD).
%This can include \textit{LD pruning}, using a program such as \texttt{PLINK} \add{CITE} to remove variants that are ``highly" correlated (e.g., pairwise-correlation $r^2 > 0.2$) with nearby variants (e.g., within a window of size ??) \add{\citep{admixture}, ADD MORE}, and/or excluding regions of the genome that are known to have extensive, long-ranging, or otherwise unusual patterns of LD \add{CITE}. 
%A list of previously-identified high LD regions is provided in Appendix ??. 
%\add{say something about how not everyone does this, it's not always clear which parameters should be used, and/or much of this work has been performed in European pop and not clear what should be done in admixed pop}


%\add{Add missing rates (SNPs and people) here? Or just frame as QC step?)}






\section{Results}

\subsection{Ancestral heterogeneity in admixed populations}

Inferred admixture proportions for three samples of African American individuals are presented in Figure \ref{fig:barplots}. 
In WHI SHARe, we compared admixture proportion estimates from a variety of model-based techniques. 
Figure \ref{fig:barplots}A presents admixture proportions estimated as genome-wide average local ancestry, using local ancestry calls from \texttt{RFMix}. 
These local ancestry based admixture proportion estimates were highly correlated (Pearson correlation $>$ 0.998) with admixture proportions from supervised and unsupervised \texttt{ADMIXTURE} analyses with two ancestral populations ($K = 2$). 
%\begin{tabular}{l|rrrr}
%& avgLA  & unsupADM    & supADM & supPrunedADM \\
%\hline
%avgLA  &    1.0000000  &   0.9984003  &   0.9983740  &    0.9970080 \\
%unsupADM    &   0.9984003  &    1.0000000  &    0.9999973   &      0.9985752 \\
%supADM   &    0.9983740   &   0.9999973  &    1.0000000  &       0.9985835 \\
%supPrunedADM  &   0.9970080  &    0.9985752  &    0.9985835      &   1.0000000\\
%\end{tabular}
In TOPMed samples, we performed unsupervised \texttt{ADMIXTURE} analyses with two ancestral populations. 
Figures \ref{fig:barplots}B and \ref{fig:barplots}C present results for unrelated African American individuals in JHS and COPDGene, respectively.
Although these analyses were unsupervised, based on prior studies of admixture in African Americans and the distribution of admixture proportions seen in WHI SHARe, we believe that the ancestral population colored dark gray in Figure \ref{fig:barplots} corresponds to European ancestry and the population colored light gray corresponds to African ancestry.
In all three samples, we observe considerable variability in the relative proportions of African and European ancestry across individuals, showing the need to adjust for global ancestry in genome-wide association studies in these admixed samples. 

%\begin{figure}
%\center
%%\includegraphics[width=\textwidth]{figs/barplots/figure1}
%\includegraphics[width=\textwidth]{figs/barplots/barplots}
%\caption{Barplots of estimated admixture proportions in (A) WHI SHARe, (B) TOPMed JHS, and (C) TOPMed COPDGene African Americans.}
%\label{fig:barplots}
%\end{figure}

\begin{figure}
\center
%\includegraphics[width=\textwidth]{figs/barplots/barplots_greyscale}
\includegraphics[width=\textwidth]{figs/finalfigs/fig1_barplots_greyscale}
\caption{Estimated admixture proportions in three samples of unrelated African American individuals. Each individual is represented by a narrow vertical bar in the plot, and the individuals are sorted by their estimated proportion of African ancestry. The admixture proportions shown here were estimated using \texttt{RFMix} in WHI SHARe (A) and an unsupervised \texttt{ADMIXTURE} analysis in TOPMed JHS (B) and COPDGene (C).}
\label{fig:barplots}
\end{figure}

%\add{
%\begin{itemize}
%\item Add WHI SHARe Hispanic Americans? (check with Tim)
%\item Possible supplemental figure: JHS and COPDGene with K = 3 (and/or K = 4 for COPDGene)
%\item Possible supplemental figure: JHS and COPDGene barplots before filtering out European Americans
%\end{itemize}
%}


\subsection{First PC captures global ancestry}



In an African American population, we might expect that only one principal component is needed to capture ancestral heterogeneity, at least with respect to differences in the relative proportion of African and European continental ancestry. 
Comparing estimated admixture proportions to principal components in WHI SHARe, JHS, and COPDGene confirms this hypothesis. 
In all three samples of African Americans, the first principal component is highly correlated with the inferred proportion of African ancestry (Pearson correlation: 0.993--0.998), while later PCs show very little correlation with genome-wide continental ancestry (Pearson correlation: -0.015--0.034).  %(see Supplemental Figure \ref{fig:pcsvsglob}).
We observe similar patterns of correlation between PCs and inferred admixture proportions regardless of the type of LD filtering (or lack thereof) performed prior to running PCA.
See Supplemental Figures \ref{fig:pcsvsglob} and \ref{fig:prunedpcsvsglob} for more detail. 

% correlation
%\begin{tabular}{l|rrrr}
%& PC1 & PC2 & PC3 & PC4 \\
%\hline
%WHI & 0.998 & 0.00181 & -0.000432 & 0.00118 \\
%JHS & 0.996 & 0.00266 & 0.0338 & 0.00782 \\
%COPD & 0.993 & 0.00291 & 0.00409 & -0.0105 
%\end{tabular}


\subsection{Later PCs may capture local genomic features}
\label{sec:CorrPlots}

%\noindent See Figure \ref{fig:corr-TOPMed} for PC-genotype correlation with naive PCs in JHS and COPDGene African Americans, and Supplemental Figure \ref{fig:corr-Eur} shows the same (except loadings instead of correlation) in COPDGene European Americans.
%
%\noindent See Figure \ref{fig:corr-compare} for a comparison of naive PCs vs exclusions vs stricter-than-default pruning vs both in WHI SHARe.
%
%\noindent See Supplemental Figures \ref{fig:corr-compare-prune}, \ref{fig:corr-compare-window}, and \ref{fig:corr-compare-exclude} for comparisons of PCs with different choices of $r^2$ thresholds (for pruning), different choices of window sizes (for pruning), and multiple iterations of data-based exclusions, respectively.


%% what do we see with no pruning
As we see above, the first principal component is highly correlated with global ancestry in these African American samples, whereas later PCs are not.
While it is possible that these higher-order principal components may be capturing sub-continental structure that is not captured by the estimated admixture proportions, we see in many cases that these later PCs are capturing local genomic features rather than genome-wide ancestry. 
This is evident from inspection of \textit{SNP loadings}, which represent the contribution of each variant to each principal component, or from investigation of the correlation between principal component scores and genotypes.

Figure \ref{fig:corr-TOPMed} presents the correlation between principal components and genotypes in JHS and COPDGene African Americans when PCs are generated without any prior LD-based pruning or filtering.
We see that variants across the genome are contributing relatively equally to the first principal component, whereas the second, third, and fourth PCs are driven by variants on a select number of chromosomes.
In JHS, for example, the second PC is particularly highly correlated with variants on chromosomes 6 and 8, and also has a higher correlation with variants on chromosomes 2, 3, and 11.
We see similar patterns, although with peaks on different combinations of chromosomes, in COPDGene (Figure \ref{fig:corr-TOPMed}B) and WHI SHARe African Americans (Figure \ref{fig:corr-compare}A).
The peaks in these genotype-PC correlation plots indicate that those principal components are primarily capturing variation at these positions along the genome rather than genome-wide global ancestry.


\begin{figure}
%\hspace{0.3in} (A) JHS \hspace{2.4in} (B) COPDGene
\center
%\includegraphics[width=\textwidth]{figs/pc_geno_corr/topmed_corr}
\includegraphics[width=\textwidth]{figs/finalfigs/fig2_topmed_corr}
\caption{Correlation between naively generated PCs (i.e., PCs that were constructed without any prior LD-based filtering or exclusions) and genotypes in JHS and COPDGene African Americans. Each panel plots the absolute value of the correlation between principal components and genotypes (on the y-axis) versus the position along the genome (x-axis).  Panels are organized vertically according to which PC is being investigated (1, 2, 3, 4) and horizontally according to the sample (A: JHS, B: COPDGene). Peaks in this plot indicate that a variant has a larger \textit{loading}, i.e., a larger contribution to that principal component.}
\label{fig:corr-TOPMed}
\end{figure}


\subsection{Impact of LD pruning}

%% why is this happening [very brief] + what happens with LD pruning
Previous authors have suggested that this phenomenon of principal components capturing local genomic features arises due to high or otherwise unusual patterns of linkage disequilibrium among variants; as a result, they recommend that variants in high LD with one another be removed prior to running PCA. 
Following these recommendations, we compare the set of principal components based on all variants to PCs generated after first removing regions of the genome known to have high LD (Table \ref{tab:highLD}), performing LD pruning, or both. 

Figure \ref{fig:corr-compare} illustrates the impact of these pre-processing steps on the correlation between genotypes and PCs in WHI SHARe African Americans. 
Panel A presents results for principal components that were generated without any prior LD-based filtering or pruning, and we see that PCs 2--4 are capturing local genomic features rather than genome-wide ancestry.
When we exclude the previously-identified high LD regions reported in Table \ref{tab:highLD} before running PCA (Figure \ref{fig:corr-compare}B), the pattern of \textit{which} SNPs are driving PCs 2--4 changes, but the issue of PCs capturing local genomic features has not been resolved. 
However, after LD pruning with an $r^2$ threshold of 0.1 and a window size of 0.5 Mb (Figure \ref{fig:corr-compare}C), we now see similar patterns with PCs 2--4 as we do with the first principal component: all variants are now contributing relatively equally to each PC. 
If we then also remove previously-identified high LD regions in addition to performing LD pruning (Figure \ref{fig:corr-compare}D), the patterns of correlation between PCs and genotypes are indistinguishable from those with LD pruning alone. 
%Similar patterns are observed in JHS and COPDGene (data not shown).

Note that the thresholds for LD pruning that we use here ($r^2 < 0.1$) are stricter than the default for many software programs and the threshold used in many studies of European populations ($r^2 < 0.2$).
If we use the larger $r^2$ threshold, we see improvement for the second and third principal components, but the fourth continues to capture local genomic features (Supplemental Figure \ref{fig:corr-compare-prune}). 

\begin{figure}
\center
%\includegraphics[width=\textwidth]{figs/WHI/pc_geno_corr/pc_geno_corr}
\includegraphics[width=\textwidth]{figs/finalfigs/fig3_pc_geno_corr}
\caption{Correlation between PCs and genotypes in WHI SHARe African Americans with different choices of pre-processing. Each panel plots the absolute value of the correlation between principal components and genotypes (on the y-axis) versus the position along the genome (x-axis).  Panels are organized vertically according to which PC is being investigated (1, 2, 3, 4) and horizontally according to the level of filtering that was applied prior to running PCA (\textit{none}: all SNPs, \textit{exclude}: after excluding regions in Table \ref{tab:highLD}, \textit{prune}: after LD pruning with an $r^2$ threshold of 0.1 and window size of 0.5 Mb, and \textit{both}: after both exclusions and LD pruning).}
\label{fig:corr-compare}
\end{figure}


%\begin{figure}[h]
%\caption{Correlation between PCs and genotypes in WHI SHARe African Americans. \\Each panel plots the absolute value of the correlation (y-axis) between principal components and genotypes at each position along the genome (x-axis). Panels are stratified according to which PC is being investigated (1, 2, 3, or 4) and what level of LD filtering was applied prior to running PCA: \textit{none} (all SNPs), \textit{exclude} (after excluding regions in Table \ref{tab:highLD}), \textit{prune} (after LD pruning with an $r^2$ threshold of 0.1 and a window size of 0.5 Mb, or \textit{both} (after exclusions and LD pruning).}
%\label{fig:corrWHI}
%\end{figure}


%\begin{figure}
%\label{fig:corrNoExclude}
%\makebox[\textwidth][c]{
%\includegraphics[width=0.24\paperwidth]{figs/JHS_prune_FALSE_1_0_0.01_snprelate_corr_1}
%\includegraphics[width=0.24\paperwidth]{figs/JHS_prune_FALSE_0.2_0.5_0.01_snprelate_corr_1}
%\includegraphics[width=0.24\paperwidth]{figs/JHS_prune_FALSE_0.1_0.5_0.01_snprelate_corr_1}
%\includegraphics[width=0.24\paperwidth]{figs/JHS_prune_FALSE_0.1_10_0.01_snprelate_corr_1}
%}
%\caption{Correlation between the first four principal components and genotypes in Jackson Heart Study African Americans when previously-identified high-LD regions (Table \ref{tab:highLD}) \textit{are not} removed prior to analysis. From left to right, panels correspond to increasingly strict levels of LD pruning: no LD pruning, LD pruning with an $r^2$ threshold of 0.2 and a window size of 0.5 Mb, LD pruning with an $r^2$ threshold of 0.1 and a window size of 0.5 Mb, and LD pruning with an $r^2$ threshold of 0.1 and a window size of 10 Mb.}
%\end{figure}

%\begin{figure}
%\label{fig:corrExclude}
%\makebox[\textwidth][c]{
%\includegraphics[width=0.24\paperwidth]{figs/JHS_prune_TRUE_1_0_0.01_snprelate_corr_1}
%\includegraphics[width=0.24\paperwidth]{figs/JHS_prune_TRUE_0.2_0.5_0.01_snprelate_corr_1}
%\includegraphics[width=0.24\paperwidth]{figs/JHS_prune_TRUE_0.1_0.5_0.01_snprelate_corr_1}
%\includegraphics[width=0.24\paperwidth]{figs/JHS_prune_TRUE_0.1_10_0.01_snprelate_corr_1}
%}
%\caption{Correlation between the first four principal components and genotypes in Jackson Heart Study African Americans when previously-identified high-LD regions (Table \ref{tab:highLD}) \textit{are} removed prior to analysis. From left to right, panels correspond to increasingly strict levels of LD pruning: no LD pruning, LD pruning with an $r^2$ threshold of 0.2 and a window size of 0.5 Mb, LD pruning with an $r^2$ threshold of 0.1 and a window size of 0.5 Mb, and LD pruning with an $r^2$ threshold of 0.1 and a window size of 10 Mb.}
%\end{figure}

%\begin{figure}[h]
%\label{fig:corrCOPDNoExclude}
%\makebox[\textwidth][c]{
%\includegraphics[width=0.24\paperwidth]{figs/COPD_prune_FALSE_1_0_0.01_snprelate_corr_1}
%\includegraphics[width=0.24\paperwidth]{figs/COPD_prune_FALSE_0.2_0.5_0.01_snprelate_corr_1}
%\includegraphics[width=0.24\paperwidth]{figs/COPD_prune_FALSE_0.1_0.5_0.01_snprelate_corr_1}
%\includegraphics[width=0.24\paperwidth]{figs/COPD_prune_FALSE_0.1_10_0.01_snprelate_corr_1}
%}
%\caption{Correlation between the first four principal components and genotypes in COPDGene African Americans when previously-identified high-LD regions (Table \ref{tab:highLD}) \textit{are not} removed prior to analysis. From left to right, panels correspond to increasingly strict levels of LD pruning: no LD pruning, LD pruning with an $r^2$ threshold of 0.2 and a window size of 0.5 Mb, LD pruning with an $r^2$ threshold of 0.1 and a window size of 0.5 Mb, and LD pruning with an $r^2$ threshold of 0.1 and a window size of 10 Mb.}
%\end{figure}

%\begin{figure}[h]
%\label{fig:corrCOPDExclude}
%\makebox[\textwidth][c]{
%\includegraphics[width=0.24\paperwidth]{figs/COPD_prune_TRUE_1_0_0.01_snprelate_corr_1}
%\includegraphics[width=0.24\paperwidth]{figs/COPD_prune_TRUE_0.2_0.5_0.01_snprelate_corr_1}
%\includegraphics[width=0.24\paperwidth]{figs/COPD_prune_TRUE_0.1_0.5_0.01_snprelate_corr_1}
%\includegraphics[width=0.24\paperwidth]{figs/COPD_prune_TRUE_0.1_10_0.01_snprelate_corr_1}
%}
%\caption{Correlation between the first four principal components and genotypes in COPDGene African Americans when previously-identified high-LD regions (Table \ref{tab:highLD}) \textit{are} removed prior to analysis. From left to right, panels correspond to increasingly strict levels of LD pruning: no LD pruning, LD pruning with an $r^2$ threshold of 0.2 and a window size of 0.5 Mb, LD pruning with an $r^2$ threshold of 0.1 and a window size of 0.5 Mb, and LD pruning with an $r^2$ threshold of 0.1 and a window size of 10 Mb.}
%\end{figure}


%\begin{figure}
%\label{fig:corrEuropeans}
%\makebox[\textwidth][c]{
%\includegraphics[width=0.32\paperwidth]{figs/EUR_prune_FALSE_1_0_0.01_snprelate_load_1}
%\includegraphics[width=0.32\paperwidth]{figs/EUR_prune_FALSE_0.2_0.5_0.01_snprelate_load_1}
%\includegraphics[width=0.32\paperwidth]{figs/EUR_prune_FALSE_0.1_0.5_0.01_snprelate_load_1}
%}
%\caption{SNP loadings for the first four principal components in COPDGene European Americans when previously-identified high-LD regions (Table \ref{tab:highLD}) are not removed prior to analysis but increasingly strict levels of LD pruning are performed. From left to right, columns correspond to no LD pruning, LD pruning with an $r^2$ threshold of 0.2 and a window size of 0.5 Mb, and LD pruning with an $r^2$ threshold of 0.1 and a window size of 0.5 Mb. \add{Change to look at correlation instead of loadings}}
%\end{figure}


%%% SKIP THIS -- write another paper on admixture mapping where this will be bigger focus
%\subsection{Confirming the importance of adjusting for population structure}
%
%\begin{itemize}
%\item show an example manhattan plot with no adjustment
%\item compare average number of spurious associations
%\item tie in theoretical results
%\end{itemize}


\subsection{Adjusting for PCs that capture local genomic features can induce spurious associations}

We have demonstrated that, especially without strict LD pruning, principal components can capture local genomic features rather than global ancestry in admixed populations.
However, it remains to be fully understood what the downstream implications would be of adjusting for these PCs in genome-wide association studies. 
We conducted a simulation study to investigate these implications further.

Figure \ref{fig:manh} presents Manhattan plots from one replicate of our simulation study.
In this setting, there is a single causal variant on chromosome 4, and we compare results from GWAS models using different ancestral heterogeneity adjustment approaches.
As expected, we see extreme inflation, i.e., statistically significant associations on \textit{every} chromosome, when we do not make any adjustment for ancestral heterogeneity (Figure \ref{fig:manh}A).
Otherwise, when we infer and adjust for ancestral heterogeneity using either PCA or estimated admixture proportions, we see a single peak in our Manhattan plot on chromosome 4---as hoped, given that is where the causal variant is located---with one notable exception.
When we adjust for the first four principal components (as has been done in previous GWAS in WHI SHARe \citep{reiner2012, carty2012}), where those PCs were generated without any prior LD-based pruning or filtering, then we see a spurious association on chromosome 6 (Figure \ref{fig:manh}C).
However, this spurious association disappears if we only adjust for the first of these PCs (Figure \ref{fig:manh}B).
Likewise, no spurious association arises if we adjust for estimated admixture proportions (Figure \ref{fig:manh}D) or if we use PCs that were generated after LD pruning and Table \ref{tab:highLD} exclusions (Figure \ref{fig:manh}, panels E and F).
Note that the causal variant, on chromosome 4, and the spurious signal, on chromosome 6, are both located in regions of the genome that are highly correlated with the PCs that were generated without any prior LD pruning (Figure \ref{fig:corr-compare}).

\begin{figure}
%\includegraphics[width=\textwidth]{figs/manhattan/WHI_manh_gwas_70}
\includegraphics[width=\textwidth]{figs/finalfigs/fig4_WHI_manh_gwas_70}
\caption{Manhattan plots from genome-wide association studies in WHI SHARe African Americans using different approaches to adjust for ancestral heterogeneity. In this example, the simulated trait depends only on the genotype at a single variant on chromosome 4. Panels present results using different adjustment approaches: (A) no adjustment; (B) one PC, with PCs calculated using all variants; (C) four PCs, with PCs calculated using all variants; (D) estimated admixture proportions; (E) one PC, with PCs calculated after LD pruning ($r^2 < 0.1$, window size = 0.5 Mb) and Table \ref{tab:highLD} exclusions; and (F) four PCs, with PCs calculated after LD pruning and exclusions. The horizontal dashed line in all panels represents the genome-wide significance threshold of $5 \times 10^{-8}$.}
\label{fig:manh}
\end{figure}

These results are not unique to this particular causal variant. 
In Figure \ref{fig:spurious}, we see that adjusting for PCs that capture local genomic features leads to higher numbers of spurious associations, on average, across all simulation settings.
Comparing models that make some sort of adjustment for ancestral heterogeneity, we observe the most spurious associations when GWAS models adjust for four principal components without any prior LD-based pruning or exclusions (represented by the orange solid line with circles in Figure \ref{fig:spurious}).
Excluding the high LD regions from Table \ref{tab:highLD} prior to running PCA (the orange solid line with triangles) reduces the number of observed spurious associations slightly, but not to the levels of the other approaches.
Given what we saw in Figure \ref{fig:corr-compare}, this is not surprising: even with these exclusions, PCs 2--4 still capture local genomic features---unless those exclusions are also combined with strict LD pruning.
On the other hand, when models only include PCs that do not capture local genomic features, the rate of spurious associations drops. 
This includes models that adjust for just the first PC (the green lines in Figure \ref{fig:spurious}) and models that include four PCs, but only after strict LD pruning (the orange dashed lines).
Models that adjust for estimated admixture proportions (the purple solid line) perform nearly identically to models that adjust for the first PC.
This, again, is not surprising, given the high correlation between admixture proportions and the first PC observed in this sample (Supplemental Figures \ref{fig:pcsvsglob} and \ref{fig:prunedpcsvsglob}).
%Given the high correlation we observed between model-based admixture proportions and the first principal component (with or without LD-based pruning or exclusions), we see very similar results for these adjustment approaches.

\begin{figure}
%\includegraphics[width=\textwidth]{figs/spurious_counts/gwas/spurious_allbeta}
\includegraphics[width=\textwidth]{figs/finalfigs/fig5_spurious_allbeta}
\caption{Comparison of the number of spurious associations in genome-wide association studies in WHI SHARe African Americans using different approaches to adjust for ancestral heterogeneity. Panel (A) displays the average number of spurious associations that were observed across all simulation settings. Remaining panels focus on the subset of simulation settings in which the causal variant has (B) a small difference in ancestral allele frequencies, (C) low SNP loadings for each of the first four PCs, (D) a high SNP loading for at least one of the first four PCs, or (D) the highest SNP loading on its chromosome for one of the first four PCs. Within each panel, we compare the number of spurious associations when GWAS models adjust for estimated admixture proportions, 1 PC (with or without LD pruning and/or Table \ref{tab:highLD} exclusions), or 4 PCs (with or without LD pruning and/or Table \ref{tab:highLD} exclusions). Results shown here are for simulated traits with a single causal variant, with effect size ($\beta$) ranging from 0 to 8.}
\label{fig:spurious}
\end{figure}

%\begin{figure}[h]
%\includegraphics[width=\textwidth]{figs/spurious_counts/gwas/figure7_spurious_beta1}
%\caption{Comparison of the number of spurious associations in genome-wide association studies in WHI SHARe African Americans using different approaches to adjust for ancestral heterogeneity. Panels display the average number of spurious associations that were observed across (A) all simulation settings, or across the subset of simulation settings in which the causal variant has (B) a small difference in ancestral allele frequencies, (C) low SNP loadings for each of the first four PCs, (D) a high SNP loading for at least one of the first four PCs, or (D) the highest SNP loading on its chromosome for one of the first four PCs. Within each panel, we compare the number of spurious associations when GWAS models adjust for model-based admixture proportions, 1 PC (with or without LD pruning and/or Table \ref{tab:highLD} exclusions), or 4 PCs (with or without LD pruning and/or Table \ref{tab:highLD} exclusions). Results shown here are for simulated traits with a single causal variant with an effect size ($\beta$) of 1. See Supplemental Figure \ref{fig:spurious-all-beta} for results with other choices of $\beta$.}
%\label{fig:spurious}
%\end{figure}




\subsection{Factors that influence the rate of spurious associations}

Our simulation results highlight various factors that influence when, and how many, spurious associations arise when adjusting for PCs that capture local genomic features.
% all approaches have no spurious assoc when there is small diff in allele freq at causal variant --- no confounding in this scenario, so really no need to adjust at all
First, we note that there are very few spurious associations, regardless of the adjustment approach (or even lack thereof), when there are small differences in ancestral allele frequencies at the causal variant (Figure \ref{fig:spurious}B). 
This is to be expected: in this scenario, the causal variant is not associated with global ancestry, so global ancestry is not a confounding variable  and there is no need for adjustment.
% low for approaches that don't capture local genomic features
Considering other simulation settings in which the causal variant has a larger difference in ancestral allele frequencies (panels C, D, and E of Figure \ref{fig:spurious}), so adjusting for ancestral heterogeneity is needed, the number of observed spurious associations remains low for models that adjust for admixture proportions, a single principal component (regardless of pre-processing), or four PCs---if those PCs were  generated after strict LD pruning.
% correlation between PC and causal variant
For the two models that adjust for PCs capturing local genomic features (i.e., the models that adjust for 4 PCs that were generated with or without Table \ref{tab:highLD} exclusions, but no LD pruning), however, we see a higher rate of spurious associations, particularly when the causal variant is highly correlated with one of those PCs.
Notably, as the size of the causal variant's SNP loading increases from low (Figure \ref{fig:spurious}C), to high (Figure \ref{fig:spurious}D), to the highest on its chromosome (Figure \ref{fig:spurious}E), we see an increasing number of spurious associations for these two approaches.
This confirms the pattern we saw in Figure \ref{fig:manh}, where a spurious association arose when we adjusted for PCs that were highly correlated with variants in several regions across the genome, and both the causal variant and spurious signal were located in one of those regions. 
% effect size of causal variant
Finally, we note that these problems worsen as the effect size of the causal variant increases.

To better understand the patterns observed in our simulation study, we compare the expected effect size estimates from GWAS models in admixed populations with two ancestral populations using different techniques for adjusting for ancestral heterogeneity.
As in our simulations, we assume that the trait depends on a single causal variant: $$y_i \stackrel{iid}{\sim} N(\beta_1 g_{i1} + \beta_\pi \pi_i, 1),$$ where $g_{i1}$ represents the number of minor alleles carried by individual $i$ at the causal variant, which we will refer to as \textit{Variant 1}, and $\pi_i$ is the individual's admixture proportion.
(Note that $\beta_\pi = 0$ in our simulation study, but we consider the more general setting here.)
We can then derive the expected effect size estimate at that causal variant, as well as a second variant that is not associated with the trait and sits on a different chromosome than the causal variant.
When we consider a GWAS model that adjusts for the true admixture proportions,
%$\pi_i$, $$E[y_i | g_{ij}, \pi_i] = \beta_0 + \beta_j g_{ij} + \beta_\pi \pi_i,$$ 
the expected effect size estimates at the causal variant (Variant 1) and the unlinked neutral variant (Variant 2) are
\begin{equation}
\begin{aligned}
E[\hat\beta_1] &= \beta_1 \\
E[\hat\beta_2] &= 0,
\end{aligned}
\label{eqn:pi}
\end{equation}
where $\beta_1$ is the true effect size of the causal variant and $\beta_2 = 0$ is the true effect size of the neutral variant.
In other words, models that perfectly adjust for ancestral heterogeneity will yield unbiased estimates of the effect size at the causal and unlinked neutral variants.
A proof of this result can be found in Supplemental Information Section \ref{sec:theory}.
%This suggests, in turn, that these models will control the rate of spurious associations.

% unadjusted model
In comparison, GWAS models that do not make any adjustment for ancestral heterogeneity will yield effect size estimates of
\begin{equation}
\begin{aligned}
E[\hat\beta_1] & = \beta_1 + \frac{(p_{11}- p_{10})V_\pi \beta_\pi }{p_{10}(1-p_{10}) + (p_{11}-p_{10})(1-p_{11}-p_{10})E_\pi + (p_{11}-p_{10})^2(V_\pi + E_\pi - E_\pi^2)} \\
E[\hat\beta_2] & = 0 + \frac{(p_{21}-p_{20}) V_\pi\{\beta_\pi + 2\beta_1(p_{11}- p_{10})\}}{p_{20}(1-p_{20}) + (p_{21}-p_{20})(1-p_{21}-p_{20})E_\pi + (p_{21}-p_{20})^2(V_\pi + E_\pi - E_\pi^2)},\\
\end{aligned}
\label{eqn:unadjusted}
\end{equation}
where $E_\pi$ and $V_\pi$ are the population mean and variance of the admixture proportions, $\beta_\pi$ is the direct effect of admixture proportions on the trait, $p_{11}, p_{10}$ are the allele frequencies of the causal variant in the two ancestral populations, and $p_{21}, p_{20}$ are the ancestral allele frequencies of the unlinked neutral variant.
(See Supplemental Information Section \ref{sec:theory} for the derivation of Equation \ref{eqn:unadjusted}.)
From these results, we see that the unadjusted model will yield a biased estimate of the effect size of the causal variant ($E[\hat\beta_1] \neq \beta_1]$) unless there is no ancestral heterogeneity (i.e., $V_\pi = 0$), global ancestry does not have a direct effect on the trait (i.e., $\beta_\pi = 0$), or the causal variant does not have different allele frequencies in the ancestral populations (i.e., $p_{11} = p_{10}$). 
We see, also, that the model can yield a biased effect size at the unlinked neutral variant ($E[\hat\beta_2] \neq 0$) even if global ancestry does not have a direct effect on the trait, provided that both the causal variant and the variant being tested have allele frequencies that differ between the two ancestral populations (i.e., $p_{11} \neq p_{10}$ and $p_{21} \neq p_{20}$).
These biased effect size estimates at neutral variants will translate into spurious associations as sample sizes increase, just as we saw in our simulations (Figures \ref{fig:manh} and \ref{fig:spurious}).
These results are summarized in Figure \ref{fig:confoundingdags} and underscore the importance of adjusting for ancestral heterogeneity even when global ancestry does not have a direct effect on the trait.

\begin{figure}
%\includegraphics[width=0.45\textwidth]{figs/confounding_dags/dag_gwas_1_v2.png}\hspace{0.1in}
%\includegraphics[width=0.55\textwidth]{figs/confounding_dags/dag_gwas_2_v2.png}
\includegraphics[width=0.45\textwidth]{figs/finalfigs/fig6a_dag_gwas_1_v2.png}\hspace{0.1in}
\includegraphics[width=0.55\textwidth]{figs/finalfigs/fig6b_dag_gwas_2_v2.png}
\caption{Directed acyclic graphs (DAGs) summarizing the conditions for confounding by global ancestry in GWAS. On the left, we see that global ancestry confounds the association at the causal variant (Variant 1) if there is ancestral heterogeneity in the population ($V_\pi > 0$), the causal variant has different allele frequencies in the ancestral populations ($p_{11} \neq p_{10}$), and global ancestry has a direct effect on the trait ($\beta_\pi \neq 0$). On the right, we see that global ancestry can confound the association at an unlinked neutral variant (Variant 2) even if global ancestry does not have a direct effect on the trait ($\beta_\pi = 0$), provided that there is ancestral heterogeneity ($V_\pi > 0$) and both the causal variant and the variant being tested have different allele frequencies in the ancestral population ($p_{11} \neq p_{10}, p_{21} \neq p_{20}$).}
\label{fig:confoundingdags}
\end{figure}


% results with two PCs
To mimic the idea of adjusting for principal components that adjust for local genomic features, we also consider a scenario in which our GWAS model adjusts for two ``principal components".
We assume that the first principal component captures global ancestry (i.e., $\mathbf{u}_1 = \boldsymbol\pi$) but the second principal component captures some feature other than global ancestry (i.e., $\mathbf{u}_2 = \mathbf{z}$ for some variable $z$).
Then, we can show (see Supplemental Information Section \ref{sec:theory}) that the expected effect size estimates at the causal variant and an unlinked neutral variant will be
\begin{equation}
\begin{aligned}
E[\hat\beta_1] & = \beta_1 \\
E[\hat\beta_2] & = 0 + \beta_1 \frac{-V_\pi E\{\text{Cov}(g_1, z \mid \pi)\} E\{\text{Cov}(g_2, z \mid \pi)\}}{V_z(V_\pi V_{g_2} - C_{g_2,\pi}^2) - V_\pi C_{g_2,z}^2 + C_{\pi, z}(2C_{g_2,\pi} C_{g_2,z} - V_{g_2}C_{\pi,z})}, \\
\end{aligned}
\label{eqn:collider}
\end{equation}
where $V_a = \text{Var}(a)$ and $C_{a,b} = \text{Cov}(a,b)$.
We see that this model adjusting for an extraneous principal component will yield an unbiased effect size estimate at the causal variant, but the same is not true for the unlinked neutral variant.
In particular, the effect size estimate at this neutral variant will be biased away from zero when there is  ancestral heterogeneity (i.e., $V_\pi \neq 0$) and the second principal component is correlated with both the causal variant and the variant being tested (i.e., $\text{Cov}(g_1, z \mid \pi) \neq 0$ and $\text{Cov}(g_2, z \mid \pi) \neq 0$)).
In other words, these results indicate that if a model adjusts for a PC that is correlated with the causal variant as well as a second variant that is not associated with the trait, then spurious associations will arise at that second neutral variant in large enough samples.
This is exactly what we observe in our simulations (Figure \ref{fig:manh}C, Figure \ref{fig:spurious}D,  Figure \ref{fig:spurious}E).
However, if the extra PC is not correlated with the causal variant, then spurious associations will not arise (Figure \ref{fig:manh}F, Figure \ref{fig:spurious}C).

% 
See Sections \ref{sec:theory} and \ref{sec:topmedsims} of the Supplemental Information for derivations and simulations validating these analytic results.

% connect to idea of collider bias
%\add{connect to collider bias --- or save for discussion?}

%% skip for now unless requested by co-author or reviewer
%how does FWER compare?
%\begin{itemize}
%\item manhattan plots for one or two simulated traits
%\item overall summary of rejection rates
%\item is it appropriate to use same significance threshold for all?
%\end{itemize}
% do they have similar power?  --- this is easier to do b/c can use existing simulations


\section{Discussion}

% explain significance of results and place them in broader context
% may contain subheadings

% summary of major takeaways
%In this paper, we compare widely-used approaches for adjusting for ancestral heterogeneity in genome-wide association studies in admixed populations.
%\add{Our work above reiterates the importance of adjusting for ancestral heterogeneity in genome-wide association studies in admixed populations and the need for careful consideration of the techniques used to make such an adjustment.
%Through theoretical work, simulation studies, and analysis of three admixed populations (WHI SHARe, TOPMed JHS, and TOPMed COPDGene African Americans) ...
%Highlights X important takeaways.}
%Of particular note, we have shown that approaches based on principal component analysis (PCA) can actually \textit{induce} spurious associations unless careful pre-processing is conducted prior to running PCA. 
%These results are particularly concerning given the wide-spread use of PCA to control for ancestral heterogeneity in GWAS.

%\noindent Need to address ancestral heterogeneity in admixed populations
We observe considerable variability in global ancestry proportions across all three admixed populations studied in this paper: the Women's Health Initiative SNP Health Association Resource (WHI SHARe), Trans-Omics for Precision Medicine Jackson Heart Study (TOPMed JHS), and TOPMed Genetic Epidemiology of Chronic Obstructive Pulmonary Disease Study (COPDGene) African Americans.
It is widely understood that adjusting for this ancestral heterogeneity in genome-wide association studies is needed in order to control for potential confounding by global ancestry and the spurious associations that can arise as a result. %but the conditions under which this confounding may arise are not fully understood.
As we've shown above, this confounding can occur even when global ancestry does not have a direct effect on the trait itself, provided that there is a causal variant elsewhere in the genome that has different allele frequencies across the ancestral populations of interest. % (Figure \ref{fig:confoundingdags}).
Although this fact has been recognized previously \citep{wacholder2002}, it is sometimes overlooked. 
Our theoretical work (Equation \ref{eqn:unadjusted}) identifies the factors that impact the magnitude of the bias incurred by GWAS models that fail to adjust for global ancestry. 
We hope that our results will serve as a reminder to researchers of the various ways in which global ancestry can confound genetic studies in admixed populations and the importance of ensuring that GWAS models appropriately adjust for ancestral heterogeneity.
%\begin{itemize}
%\item we observe considerable heterogeneity in global ancestry proportions in admixed populations studied here, as in other studies 
%\item well-established that global ancestry is a potential confounding variable
%\item this confounding can exist even if global ancestry does not have a direct effect on the trait (as demonstrated by our simulation studies and theoretical results)
%\item $\rightarrow$ important to carefully measure and adjust for ancestral heterogeneity in GWAS in admixed populations
%\end{itemize}

%\noindent Comparing (naive) PCs and admixture proportions
A common approach for adjusting for ancestral heterogeneity in GWAS involves including global ancestry as a covariate in marginal regression models, with global ancestry estimated using either model-based approaches or principal component analysis.
In WHI SHARe, TOPMed JHS, and TOPMed COPDGene African Americans, the first principal component is highly correlated with estimates of the genome-wide proportion of African ancestry %(Figure \ref{fig:pcsvsglob}) 
and models adjusting for either perform similarly.
Later PCs, however, do not correlate with global ancestry and models that adjust for these PCs anyway---a common practice in the literature---can yield elevated rates of spurious associations, particularly when those PCs capture local genomic features rather than genome-wide ancestry.  %(Figure \ref{fig:spurious} and \ref{fig:spurious-all-beta}. 
Prior work (e.g., \citep{zou2010, prive2020}) has shown that PCs can detect regions with high, extensive, or otherwise unusual patterns of linkage disequilibrium (Table \ref{tab:highLD}), but the patterns observed in those studies---primarily involving individuals of European ancestry---differ from what we see here.
These prior studies typically saw PCs that were driven by variants in a single high-LD region, just as we see in TOPMed COPDGene European Americans (Supplemental Figure \ref{fig:corr-Eur}).
However, in all three of the admixed samples that we investigated, we see instead that PCs are often correlated with variants in \textit{multiple} regions, across multiple chromosomes. 
This has important downstream implications.
%\begin{itemize}
%\item both widely used for measuring and adjusting for ancestral heterogeneity
%\item in AA, first PC correlated with global ancestry but later PCs are not (in HL: TBD)
%\item instead, later PCs often capture local genomic features (e.g., regions with extensive LD)
%\item while this has been documented before, note that, in contrast to what has been observed in EUR, we see that PCs seem to capture SNPs on more than one chromosome (multiple peaks in SNP loading plots); whereas in EUR we often just see one peak (cite Zou, Prive, other examples?) $\rightarrow$ why? LD patterns in admixed pops differ from those in EUR %% in discussion, note how this differs from what had previously been seen in Europeans (one peak vs multiple peaks, weaker pruning); refer to COPD example in supplement
%\end{itemize}

%\noindent spurious assoc & collider bias
Our simulations show that GWAS models adjusting for PCs that capture multiple local genomic features have elevated rates of spurious associations. %, with the number of spurious associations increasing with the effect size of the causal variant and the strength of the correlation between that causal variant and the PC by which it is captured.
This can be explained by the concept of \textit{collider bias} (Figure \ref{fig:collider}).
If a PC is correlated with the genotype of multiple variants, and at least one of those variants is associated with the trait, then the PC becomes a \textit{collider variable} when testing the association between the other variants and the trait.
Adjusting for collider variables can induce a spurious association between variables that are otherwise unlinked \citep{elwert2014}.
This is precisely what we see above.
Our theoretical work (Equation \ref{eqn:collider}) shows that GWAS models adjusting for an extraneous PC will yield biased estimates of variant effect sizes, with the magnitude of that bias increasing with the effect size of the causal variant, the strength of the correlation between the PC and the variants it captures, and the amount of ancestral heterogeneity within the population.
When that bias is large enough, it can lead to a spurious association---as our simulations show.
Given the similarities in terms of which regions of the genome tend to be correlated with PCs across different admixed populations (see Figures \ref{fig:corr-TOPMed} and \ref{fig:corr-compare}, for example), it is possible that these spurious associations may even replicate across studies.  
%\add{\begin{itemize}
%\item adjusting for these PCs can lead to spurious associations
%\item this is due to a phenomenon known as collider bias
%\item what do theory and sims tell us about when/how likely a spurious association is to occur? 
%\item could spurious associations replicate? (given that peaks often occur in similar places across datasets)
%\item prior work on collider bias in GWAS
%\end{itemize}}

%% add: Zou et al show that including PCs that capture LD can lead to reduced power within the regions captured

\begin{figure}
\center
%\includegraphics[width=0.5\textwidth]{figs/confounding_dags/dag_collider.png}
\includegraphics[width=0.5\textwidth]{figs/finalfigs/fig7_dag_collider.png}
\caption{Collider bias in GWAS. Suppose that, instead of genome-wide global ancestry ($\pi$), a principal component ($z$) captures the genotype of two variants ($g_1, g_2$). If one of those variants ($g_1$) is associated with the trait ($y$), then the PC will be a collider variable---rather than a confounding variable---when testing the association between the other variant ($g_2$) and the trait. Adjusting for the PC can then induce a spurious association between the trait and the neutral variant as a result of collider bias.}
\label{fig:collider}
\end{figure}

%\noindent prior work on collider bias, how many PCs to include
Increasing attention has been paid to the issue of collider bias in genetic association studies in recent years \citep{aschard2015, day2016, cai2022, hemani2022}, but to our knowledge this is the first paper to fully demonstrate the concerns related to including extraneous principal components in GWAS models.
In fact, the focus in the literature has typically been on demonstrating issues that can arise when adjusting for \textit{too few} principal components \citep{eigenstrat, kang2010, Yao2022}. %%says "Our tests also found PCA robust to large numbers of PCs, far beyond the optimal choice, agreeing with previous anecdotal observations (Price et al., 2006 "PCA corrects for strafication..."; Kang et al., 2010 "Variance component model to account...") in contrast to using too few PCs for which there is a large performance penalty."
One recent paper does show that adjusting for principal components can lead to replicating spurious associations in gene expression studies due to collider bias \citep{dahl2019}.
As a short aside, the authors mention that collider bias could arise in GWAS, but they imply that the magnitude of this bias would be small--- and thus not of practical concern---given that ``causal SNPs have much lower leverage on genetic PCs" (i.e., the correlation between PCs and causal variants is small).  
Our findings support this conclusion that the magnitude of collider bias depends on the strength of the correlation between the PCs and the variants they capture.
However, we see that this correlation between PCs and genetic variants can be non-trivial in admixed populations (without careful pre-processing prior to running PCA), implying that collider bias is of more concern in this setting than previous authors may have realized.
We also see that the amount of bias depends on the variance of admixture proportions across a population (denoted $V_\pi$ in Equation \ref{eqn:collider}), which could explain why work that has focused on more ancestrally homogenous populations may not have identified this issue previously.
That said, it is important to raise the question of the magnitude of bias that could be expected in a ``typical" genome-wide association study. 
The smaller the effect size of the causal variant, the smaller the number of spurious associations that we observed in our simulation study (Figure \ref{fig:spurious}), and when studying complex traits/diseases we would expect the causal variant effect sizes to be fairly small.
However, in these settings we would also expect that there are \textit{multiple} causal variants---not just a single causal variant, as we assumed for the sake of simplicity in our simulations and theoretical work---and then these effects may be aggregated \citep{di2011}. 
In any case, it is worth considering what steps can be taken to reduce or eliminate concerns about potential collider bias altogether.
%\add{\begin{itemize}
%\item summarize gene expression paper (dissertation ref 146)
%\item explain why more concerning in admixed pops (connect to theory)
%\item limitation: what's a realistic effect size, multiple causal variants
%\end{itemize}}

%\noindent possible solutions: Impact of LD pruning and removing high LD regions
In our analysis of genotype and sequence data from unrelated WHI SHARe, TOPMed JHS, and TOPMed COPDGene African Americans, we found that all but the first principal component were  correlated with small regions of the genome---and thus have the potential to be collider variables---unless careful pre-processing of genotype data was performed prior to running PCA.
As mentioned earlier, previous studies have found that PCs can be driven primarily by small regions of the genome, and as a result have suggested that these regions be excluded (Table \ref{tab:highLD}) and/or that LD pruning be performed prior to running PCA.
However, the motivation for this LD-based filtering has typically been framed in terms of the ability of the principal components to capture global ancestry, as well as the computational complexity of running PCA, rather than the downstream implications on association testing that we have highlighted here.
Furthermore, we have found that excluding the regions listed in Table \ref{tab:highLD}, without also performing LD pruning, does not seem to solve these issues in admixed populations: we still see peaks in the PC-genotype correlation plots and models adjusting for these PCs continue to show elevated rates of spurious associations.
A more tedious iterative approach of identifying  and removing potentially problematic regions based on our own data also does not prevent PCs from capturing local genomic features within a reasonable number of iterations (Supplemental Figure \ref{fig:corr-compare-exclude}).
LD pruning, on the other hand, proves to be more successful, at least when looking at the first four principal components in WHI SHARe African Americans.
Our results highlight that a stricter threshold (e.g., $r^2 = 0.1$) may be needed for LD pruning in admixed populations than the $r^2 = 0.2$ threshold that is often suggested in the literature: see Supplemental Information Section \ref{sec:comparefilters} for a comparison.
Given that linkage disequilibrium patterns differ between admixed populations and the European populations upon which much of this prior work was based, %\add{CITE 147, 148}
it is not surprising that different LD-based filtering techniques are required here.

Our analyses in this paper have focused on samples of African American individuals, but we believe that our findings generalize more broadly.
In any sample, it is important to check SNP loadings or the correlation between PCs and genotypes.
If multiple variants are associated with a PC and sample size is sufficiently large, spurious associations can arise.  
In some samples, it may be necessary to consider even stricter LD pruning (e.g., smaller $r^2$ threshold, larger window size), to include fewer PCs, or to consider an alternate approach altogether (e.g., adjusting for estimated admixture proportions).

%\add{\begin{itemize}
%\item after LD pruning, PCs no longer exhibit patterns of being driven by select few SNPs (at least for PCs 2-4)
%\item note that we had to use smaller $r^2$ than often recommended in literature for this to be true (refer to supplemental figure) $\rightarrow$ why? LD patterns in admixed pops differ from those in EUR 
%\item excluding previously-identified high LD regions doesn't seem to be as effective $\rightarrow$ why? LD patterns in admixed pops differ from those in EUR
%\item mention data-based exclusions approach (add supplemental figure)
%\item ADD??? even strict LD Pruning doesn't seem to remove all correlation between SNPs and genotypes (e.g., later PCs in WHI, TOPMed COPDGene)
%\end{itemize}}

%\noindent Conclusions & Recommendations
Ultimately, our work demonstrates the challenges that can arise in appropriately adjusting for ancestral heterogeneity in genome-wide association studies in admixed populations.
For populations where we have a good idea of the number of ancestral populations of interest and relevant reference panel data is readily available, GWAS models adjusting for estimated global ancestry proportions rather than principal components perform well. 
PCA can offer advantages over model-based ancestry inference methods, but careful consideration must be given to how many PCs should be included and what those PCs are capturing. 
In the African American samples studied here, for example, a single principal component was sufficient for controlling spurious associations induced by population structure. 
In some simulation settings, we did see a small drop in the number of spurious associations when we included three additional PCs, but only after strict LD pruning.
Careful pre-processing of data prior to running PCA, combined with thorough diagnostics (e.g., by calculating and plotting SNP loadings or the correlation between PCs and genotypes), is critical to ensure that models do not include principal components that could cause the very problem---spurious associations---that these techniques aim to solve.
%\begin{itemize}
%\item If using PCs, carefully inspect SNP loadings and/or correlation between PCs and genotypes
%\item If using PCs, don't use more than you need
%\item Consider using global ancestry proportions (although further work is needed to reliably capture sub-continental structure)
%\end{itemize}



% future work: 
%% GWAS vs admixture mapping
%% sequence vs genotype data
%% how does this apply to mixed models?



\newpage

\section{Supplemental Information}

% briefly list what types of data are included in supplemental data (a separate PDF document)
% e.g., "Supplemental Data include four figures and two tables"

Supplemental Information includes 13 figures, as well as proofs and simulations validating the theoretical results presented in the main paper.


\section{Declaration of Interests}

%We ask that you and all authors disclose any personal financial interests (examples include stocks or shares in companies with interests related to the submitted work or consulting fees from companies that could have interests related to the work), professional affiliations, advisory positions, board memberships, or patent holdings that are related to the subject matter of the contribution. As a guideline, you need to declare an interest for (1) any affiliation associated with a payment or financial benefit exceeding \$10,000 p.a. or 5\% ownership of a company or (2) research funding by a company with related interests. You do not need to disclose diversified mutual funds, 401ks, or investment trusts. 
%More info here: https://www.cell.com/declaration-of-interests.
%\edit{If you have anything to disclose (see \href{https://www.cell.com/declaration-of-interests}{https://www.cell.com/declaration-of-interests}), please let me know.}

\noindent The authors declare no competing interests.




\section{Acknowledgments}

% contributions from nonauthors
% list funding sources
% add any necessary data-related acknowledgments

% WHI
The WHI program is funded by the National Heart, Lung, and Blood Institute, National Institutes of Health, U.S. Department of Health and Human Services through contracts HHSN268201600018C, HHSN268201600001C, HHSN268201600002C, HHSN268201600003C, and HHSN268201600004C. The authors thank the WHI investigators and staff for their dedication, and the study participants for making the program
possible. A listing of WHI investigators can be found at \href{https://www.whi.org/doc/WHIInvestigator-Long-List.pdf}{https://www.whi.org/doc/WHIInvestigator-Long-List.pdf}.

% TOPMed
Molecular data for the Trans-Omics in Precision Medicine (TOPMed) program was supported by the National Heart, Lung, and Blood Institute (NHLBI). 
Core support including centralized genomic-read mapping and genotype calling along with variant quality metrics and filtering were provided by the TOPMed Informatics Research Center (3R01HL-117626-02S1; contract HHSN268201800002I). 
Core support including phenotype harmonization, data management, sample identity QC, and general program coordination were provided by the TOPMed Data Coordinating Center (R01HL-120393; U01HL-120393; contract HHSN268201800001I). 
We gratefully acknowledge the studies and participants who provided biological samples and data for TOPMed. 
The Jackson Heart Study is supported and conducted in collaboration with Jackson State University (HHSN268201300049C and HHSN268201300050C), Tougaloo College (HHSN268201300048C), and the University of Mississippi Medical Center (HHSN268201300046C and HHSN268201300047C) contracts from NHLBI and the National Institute for Minority Health and Health Disparities (NIMHD); genome sequencing was funded by HHSN268201100037C. 
The COPDGene study was supported by NIH grants U01 HL089856 and U01 HL089897. The COPDGene project is also supported by the COPD Foundation through contributions made by an Industry Advisory Board comprised of Pfizer, AstraZeneca, Boehringer Ingelheim, Novartis, and Sunovion.

% grants
K.E.G. was supported in part by the National Science Foundation Graduate Research Fellowship Program under grant no. DGE-1256082. Any opinions, findings, and conclusions or recommendations expressed in this material are those of the author(s) and do not necessarily reflect the views of the National Science Foundation.
S.R.B. and B.L.B. were supported by the National Human Genome Research Institute of the National Institutes of Health under award number HG010869. The content is solely the responsibility of the authors and does not necessarily represent the official views of the National Institutes of Health.

%\noindent \edit{If there is anything else you would like me to add, please let me know.}




\section{Data and Code Availability}

%statement describing the availability of new datasets and/or code associated with the paper.
%includes any conditions for access of datasets and/or code not publicly available.
%also include any accession numbers, DOIs or unique identifiers, or web links to deposited datasets

%%examples
%The [datasets/code] generated during this study are available at [name of repository] [accession code/web link].
%The published article includes all [datasets/code] generated or analyzed during this study.
%This study did not generate/analyze [datasets/code].
%There are restrictions to the availability of [dataset/code] due to [reason for restrictions].
%Original/source data for [figures/datatype] in the paper is available [e.g., Mendeley Data DOI].
%The [datasets/code] supporting the current study have not been deposited in a public repository because [reason data are not public] but are available from the corresponding author on request.

WHI SHARe genotype data is available on dbGaP (accession number: phs000386) or directly through WHI according to the policy outlined at \href{https://www.whi.org/doc/WHI-genetic-data-transfer-policy.pdf}{https://www.whi.org/doc/WHI-genetic-data-transfer-policy.pdf}. 
TOPMed whole genome sequence data is also available from dbGaP (Jackson Heart Study: phs000964, Genetic Epidemiology of Chronic Pulmonary Disease Study: phs000951). 
All software packages used throughout this paper are freely available online: 

%Data: 
%\begin{itemize}
%\item WHI Data: \add{TBD}
%\item TOPMed Sequence Data: \href{https://www.ncbi.nlm.nih.gov/gap/}{https://www.ncbi.nlm.nih.gov/gap/}
%\end{itemize}

\begin{itemize}
\item bcftools \citep{bcftools} (quality control): \href{https://samtools.github.io/bcftools/}{https://samtools.github.io/bcftools/}
%\item Beagle (phasing and imputation): 
\item RFMix \citep{rfmix} (local ancestry inference):  \href{https://sites.google.com/site/rfmixlocalancestryinference/}{https://sites.google.com/site/rfmixlocalancestryinference/}
\item ADMIXTURE \citep{admixture} (global ancestry inference): \href{https://dalexander.github.io/admixture/}{https://dalexander.github.io/admixture/}
\item PCRelate \citep{conomos2016related} and PC-AiR \citep{conomos2015}
 (identifying unrelated individuals): \href{https://rdrr.io/bioc/GENESIS/}{https://rdrr.io/bioc/GENESIS/}
\item SNPRelate \citep{snprelate} (LD pruning, PCA, and PCA-related diagnostics): \\ \href{https://www.bioconductor.org/packages/release/bioc/html/SNPRelate.html}{https://www.bioconductor.org/packages/release/bioc/html/SNPRelate.html} 
\item PLINK \citep{plink} (GWAS): \href{https://zzz.bwh.harvard.edu/plink/}{https://zzz.bwh.harvard.edu/plink/}
\item TOPMed Analysis Pipeline (identifying unrelated individuals, LD pruning, PCA,  PCA-related diagnostics, and association studies in whole genome sequence data): \\ \href{https://github.com/UW-GAC/analysis_pipeline}{https://github.com/UW-GAC/analysis\_pipeline}
\item R (analyzing and visualizing results): \href{https://cran.r-project.org/}{https://cran.r-project.org/}
\end{itemize}

\noindent Other resources pertaining to this paper, including download-able lists of the high LD regions in Table \ref{tab:highLD} in various builds, can be found on the lead author's GitHub page: 
\begin{itemize}
\item \href{https://github.com/kegrinde/PCA}{https://github.com/kegrinde/PCA}.
\end{itemize}

%\noindent \add{GitHub Repo will include: lists of regions to exclude, code for LD pruning, excluding, and plotting loadings}


\newpage
\section{References}

%include only articles that are published.
%"et al." should be used only after 10 authors.
%Please use the following styles for references:

%%Article in a periodical
%1. Leach, N.T., Sun, Y.,Michaud, S., Zheng, Y., Ligon, K.L., Ligon, A.H., Sander, T., Korf, B.R., Lu, W., Harris, D.J., et al. (2007). Disruption of diacylglycerol kinase delta (DGKD) associated with seizures in humans and mice. Am. J. Hum. Genet. 80, 792–799.

%%Article in a book
%2. King, S.M. (2003). Dynein motors: Structure, mechanochemistry and regulation. In Molecular Motors, M. Schliwa, ed. (Wiley-VCH Verlag GmbH), pp. 45–78.

%%Entire book
%3. Cowan, W.M., Jessell, T.M., and Zipursky, S.L. (1997). Molecular and Cellular Approaches to Neural Development (Oxford University Press).

%%Online reference
%4. Rothwarf, D.M., and Karin, M. (1999). The NF-kB pathway: a paradigm in information transfer from membrane to nucleus. Science’s STKE, http://www.stke.org/cgi/content/full/OC_sigtrans;1999/5/rel.

%%Computer program
%5. Hubbard, S.J. and Thornton, J.M. (1993). NACCESS computer program (Department of Biochemistry and Molecular Biology, University College London).
% Software may also be cited in text; for an in-text citation, include the name of the manufacturer in parentheses.

%%Dissertation/Thesis
%6. Smith, J.P. (1985). DNA sequences. PhD thesis (Massachusetts Institute of Technology).

%The current AJHG reference format for Endnote users can be downloaded from http://www.endnote.com/support/enstyles.asp
%The current AJHG reference format for RefMan users can be downloaded from http://www.refman.com/support/rmstyles.asp

%%In-Text Citations
%Unpublished data, abstracts, and personal communications may be cited within the text only. Submitted articles that have not yet been accepted should be cited as data not shown, unpublished data, or a personal communication.
%Unpublished data may refer only to work from an author of the manuscript being submitted.
%A personal communication should be documented by a letter of permission (this may be in the form of an e-mail communication, letter, or other appropriate form of permission).
%Please use the following style for such citations:

%%Unpublished data 
%(M.A., unpublished data)

%%Abstract
%(M. Adams et al., 1997, Soc. Neurosci., abstract)

%%Personal communication
%(M. Adams, personal communication) 

\bibliographystyle{ajhg}
\bibliography{spurious}


\newpage
%\section{Figure Titles and Legends}
%
%\add{for AJHG submission, will need to move all figures here}

% brief title describing entire figure without citing specific panels
% subsequent description of each panel

% numbered consecutively with whole numbers (e.g., Figure 1, Figure 2, etc. rather than Figure 1a, Figure 1b, etc.)

% figures may not exceed one page

% figure titles may not contain parenthetical information, reference citations, or footnotes

% all reference citations within a figure must also be included in the figure legend

%For any figures presenting pooled data, the measures should be defined in the figure legends (for example, data are represented as the mean ± SEM).

%%https://www.cell.com/figureguidelines


%\section{Tables}
%
%\add{for AJHG submission, will need to move all tables here}


%Include tables in the submitted manuscript after the figure titles and legends. Tables should not be saved as figures, i.e., as .jpg or .tif files. All tables intended for print should be incorporated into the end of the manuscript Word file. Tables should not be uploaded individually.

%When creating a table, please use the Microsoft Word table function, and please do not place an Excel table into a Word document. Tables not created with the Microsoft Word table function will be sent back for revision. Do not submit a table in PDF format.

%Word tables should not be tab or space delineatedand and should not include colored text or shading, but embedded graphics with color are OK.

%Do not use paragraph returns to separate data within a cell.

%Tables should include a title, and footnotes and/or legends should be concise. 

%Table titles may not contain parenthetical information, reference citations, or footnote citations.

%Use superscripted lowercase letters (beginning with “a”) for footnotes in tables. Do not use numbers or symbols.

%Tables must be numbered as Table 1, Table 2, Table 3, etc., rather than as Table 1a, Table 1b, Table 1c, etc.

%If italic font is used within a table to indicate some feature of the data, an explanation of its meaning must be given in the table legend. Bold text may not be used in tables.

%If a referenced paper or study is mentioned within a table, it must be included in the References list and must be followed by its appropriate citation number (e.g., “Author et al.1”) within the table.

%All abbreviations within a table must be defined in the table legend or footnotes.





\end{document}